\documentclass[a4paper,11pt]{article}
\usepackage[textwidth=35em]{geometry}
\usepackage{calc,xeCJK,xeCJKfntef,fontspec,microtype,datetime2,paralist,enumitem}

\setmainfont{CMU Serif}
\setromanfont{CMU Serif}
\setsansfont{TeX Gyre Heros}
\setmonofont{JetBrains Mono NL}

\setCJKmainfont{Noto Serif CJK SC}
\setCJKromanfont{Noto Serif CJK SC}
\setCJKsansfont{Noto Sans CJK SC}
\setCJKmonofont{Noto Sans CJK SC}

\setlength{\parskip}{7pt}
\setlength{\parindent}{0pt}
\linespread{1.1}
\frenchspacing


\title{特殊有限合伙框架协议\\Special Limited Partnership\\Framework Agreement}
\author{NERUTHES}
\date{\today}


\begin{document}
\maketitle




\section{前言}
\begin{enumerate}
    \item 在本协议的框架内,1名\CJKunderline{一般合伙人}和5名\CJKunderline{有限合伙人}形成\CJKunderline{特殊有限合伙}关系,
    名为\CJKunderline{Enter Name Here Studio}(以下称为合伙产物),不设独立法人实体,寄宿在一般合伙人的人格上。
    \item 一般合伙人:Some Hosting Vehicle Inc(法人)。
    \item 有限合伙人:
    \begin{enumerate}
        \item 张三(自然人)
        \item 李四(自然人)
        \item 王五(自然人)
        \item 刘太能(自然人)
        \item 曹多多(自然人)
    \end{enumerate}
\end{enumerate}



\section{治理}
\begin{enumerate}
    \item 一般合伙人的权责参照中华人民和共和国公司法定义的董事长、法定代表人。
    \item 有限合伙人的权责参照中华人民和共和国公司法定义的小股东。
    \item 全体有限合伙人可以选举最多2名监察员,其权责参照中华人民和共和国公司法定义的监事。
    \item 表决权的40\%固定属于一般合伙人。全体合伙人按照等效实缴资本的比例分享其余表决权。
    \item 一般合伙人应编制财务报表,向全体合伙人公示报表和原始财务信息,接受监督。
\end{enumerate}



\section{出资}
\begin{enumerate}
    \item 一般合伙人全额垫付合伙产物的经营开支,其金额视为等效实缴资本。
    \item 有限合伙人可以主动出资,但出资后的等效实缴资本总额不得超过一般合伙人的等效实缴资本总额。
    \item 经全体合伙人同意,有限合伙人可以以劳动形式出资,设置一个名义月薪,劳动贡献视为等效实缴资本。
    \item 一般合伙人为法人的,可以安排其法定代表人参与劳动,享受与有限合伙人相同的以劳动形式出资的权利,产生的等效实缴资本属于一般合伙人,但名义月薪不超过其他有限合伙人。
    \item 经全体合伙人同意,合伙人的劳动成果带来特大利益的,可以专门授予其一定数额的等效实缴资本。
\end{enumerate}



\section{撤资}
\begin{enumerate}
    \item 有限合伙人可以主动撤资。其历史累计贡献的等效实缴资本由一般合伙人买断,实现股转债。
    \item 一般合伙人和有限合伙人可以对有限合伙人的等效实缴资本发起买断,溢价率至少为2倍(例如,等效实缴资本为1万元的,出价至少为3万元)。
\end{enumerate}



\section{形式变更}
\begin{enumerate}
    \item 过半合伙人同意的,合伙产物应将组织形式变更为普通商业公司,重新制定章程,实现正规注册登记。
\end{enumerate}



\section{利益分配}
\begin{enumerate}
    \item 每个季度结束时结算本期利润。本期利润扣除合理扣除项后为正的,至少90\%用于增加所有者权益。
    \item 合理扣除项包括:计提偿债准备金、填平往年亏损。
    \item 合伙产物应积极分红。未分配利润、资本公积、所有者权益科目余额总和不超过15万元。
\end{enumerate}




\end{document}
