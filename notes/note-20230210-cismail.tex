\documentclass[a4paper,11pt]{article}
\usepackage{tocloft,tabu}

% Use letterhead1 library
\input{/home/neruthes/DEV/ntexlibs/lib/letterhead1.tex}

\usepackage{amssymb}

\usepackage{paralist,enumerate,enumitem}





% Fonts
\setmainfont{NewComputerModern}
\setromanfont{NewComputerModern}
\setsansfont{Inter}
\setmonofont{JetBrains Mono NL}
\setCJKmainfont{FandolSong}
\setCJKromanfont{FandolSong}
\setCJKsansfont{Noto Sans CJK SC}
\setCJKmonofont{Noto Sans CJK SC}

% Use alternative geometry
\renewcommand{\letterfooterwidth}[0]{44em}

\newcommand{\refreshgeometry}[0]{
    \newgeometry{a4paper,textwidth=\letterfooterwidth,tmargin=45mm,bmargin=29mm}
}
\newcommand{\fontsizepteleven}[0]{
    \renewcommand{\letterfooterwidth}[0]{40em}
    \refreshgeometry
}
\newcommand{\fontsizeptwelve}[0]{
    \renewcommand{\letterfooterwidth}[0]{37em}
    \refreshgeometry
}

% % Some dimensions
\setlength{\tabulinesep}{3pt}
% \setlength{\parindent}{2em}
% \setlength{\parskip}{4pt}

\newcommand{\simpledoctitle}[0]{%
    \begin{minipage}{\linewidth}%
        \parskip=6pt%
        \center%
        \rmfamily%
        \fontspec[Ligatures=TeX]{Latin Modern Roman}%
        {\LARGE\cacheddoctitlebig}\par\vspace{4pt}%
    \end{minipage}\par
    \vspace{15pt}\hrule%
    \vspace{15pt}%
    \renewcommand{\ifootcornerL}[0]{\cacheddoctitlefooter}%
}
\newcommand{\fulldoctitle}[0]{%
    \begin{minipage}{\linewidth}%
        \parskip=6pt%
        \center%
        \rmfamily%
        \fontspec[Ligatures=TeX]{Latin Modern Roman}%
        {\LARGE\cacheddoctitlebig}\par\vspace{4pt}%
        {\sffamily\normalsize\cacheddocauthor}\par%
        {\sffamily\normalsize\cacheddocdate}%
    \end{minipage}\par
    \vspace{15pt}\hrule%
    \vspace{15pt}%
    \renewcommand{\ifootcornerL}[0]{\cacheddoctitlefooter}%
}
\newcommand{\setrealdisplaytitle}[1]{
    \renewcommand{\cacheddoctitlebig}[0]{#1}
    \renewcommand{\cacheddoctitlefooter}[0]{#1}
}
\newcommand{\Nmaketoc}[0]{\vspace{20pt}\hrule\tableofcontents\vspace{20pt}\hrule}


\renewcommand{\letterfooterfont}[0]{\sffamily\footnotesize}
\renewcommand{\ifootcornerL}[0]{A Note by \theauthor}
\renewcommand{\ifootcornerR}[0]{Page \thepage}

\newcommand{\icode}[1]{\texttt{\footnotesize#1}}



\newcommand{\Asection}[1]{
    \section*{#1}
    \addcontentsline{toc}{section}{#1}
}
\newcommand{\Asubsection}[1]{
    \subsection*{#1}
    \addcontentsline{toc}{subsection}{#1}
}



\fontsizepteleven
% \fontsizeptwelve
\docsetup{Cismail Protocol Specification Version 1}{Neruthes}{2023-02-12 (WIP)}
\linespread{1.1}







\begin{document}
% \raggedright

\fulldoctitle


\section*{Abstract}

This document describes the design specification of the Cismail protocol.
``CIS'' in Cismail is short for cryptographically innovative-and-secure.

This is an attempt to put the 7-layer model%
\footnote{(WIP) 2023. Neruthes. Model of Seven Layers for Cryptographic Communication Protocols. [\href{https://pub-714f8d634e8f451d9f2fe91a4debfa23.r2.dev/keep/ntexdb/note-20230206-crypt-suite.pdf--6c7cbbc319948e0dd9e2ae6b5fcdfa27.pdf}{PDF link}]}
(of cryptographic communication protocols) in real use.

Cismail is a minimalist protocol which is designed to offer efficient cryptographic protection
with a workflow similar to existing practices with the OpenPGP protocol (RFC 4880).
It serves similar scenarios.

At this stage, the Cismail protocol is simple and straightforward.
It allows a plaintext message to be encrypted for a recipient from an author with deniable authentication.




\section{Sample Workflow}

\subsection{Background}

Alice wants to write to Bob.
% Alice has 2 keypairs;
% one for signature ($AS.priv$, $AS.pub$),
% and the other for encryption ($AE.priv$, $AE.pub$).
% The same applies to Bob ($BS.priv$, $BS.pub$, $BE.priv$, $BE.pub$).

They have exchanged their public keys (self-signed X.509 certificates) already before this story begins.

\subsection{Writing Message}

All Alice wants to say is a simple phrase consisting of 5 words: Lorem ipsum dolor sit amet.

Therefore, Alice has a UTF-8 plaintext raw message:

\begin{lstlisting}
Lorem ipsum dolor sit amet.
\end{lstlisting}

\subsection{Message Packaging Layer: Basic Mail Message Markup (BMMM)}

Alice needs to wrap the raw message into a mail payload by adding a header,
which consists of a few LF-delimited data fields and an ending mark (a line consisting of exactly 10 ``-'' characters).
These data fields allow the recipient to understand the context of the raw text.

\begin{lstlisting}
p: bmmm-1.0
app: cismail-1.0
from: Alice
to: Bob
from_key: <X.509 Subject Key Identifier (hex)>
to_key: <X.509 Subject Key Identifier (hex)>
markup: plain
garbage_bytes: <Some random bytes ...>
----------

Lorem ipsum dolor sit amet.
\end{lstlisting}

The definitions of the data fields in the header will be discussed later.

Note: Random garbage bytes are inserted to make sure that the length (in bytes) of the output is always a multiple of 256.

\subsection{Session Management Layer: Null Session Protocol}

Null Session Protocol is a really simple Session Management Layer protocol.
It does nothing at all.
It maintains no state and adds no prefix.
It terminates after finishes echoing the first message going into it.

\subsection{Channel Management Layer: One-Way ECDH-Hybrid Temporary Channel Protocol}

\subsubsection{Key Negotiation}

Alice generates an ephemeral keypair ($Ephe.priv$, $Ephe.pub$).
The ephemeral keypair is used in an ECDH operation to calculate the ephemeral shared secret `$Ess$'.

\begin{displaymath}
	Ess = ECDH~(~Ephe.pri~,~BE.pub~)
\end{displaymath}

Alice calculates the permanent shared secret `$Pss$' between them.

\begin{displaymath}
	Pss = ECDH~(~AE.pri~,~BE.pub~)
\end{displaymath}

Let the predefined magic salt number `$Salt$' be dee44fb4 c92dd378 c7d948a1 c76b19e9 e6e16e9e fbbdd087 66922b93 c19f1437.

Alice calculates the real cipher key `$Cipher$' by passing the data as parameters to PBKDF2.

\begin{center}
	\begin{tabu}{ll}
		\toprule
		Parameter  & Value          \\
		\midrule
		PRF        & HMAC (SHA-256) \\
		Salt       & $Salt$         \\
		Password   & $Ess$          \\
		Iterations & 207777         \\
		dkLen      & 256            \\
		\bottomrule
	\end{tabu}
\end{center}

\subsubsection{Symmetric Encryption}

Alice generates a random IV `$V$'.

Alice uses the parameters to encrypt the session member payload `$SMP$' from the previous stage to get the encrypted blob `$EB$'.
The encryption should be compatible with OpenSSL command line operation.

\begin{displaymath}
	EB = AES\_256\_CFB.encrypt(~IV~,~Cipher~,~SMP~)
\end{displaymath}

\subsubsection{Deniable Authentication}

Alice calculates the HMAC (with SHA-256) `$MAC$' for `$SMP$'.

\begin{displaymath}
	MAC = HMAC(~Pss~,~HMAC(~Ess~,~SMP~)~)
\end{displaymath}

With the HMAC value `$MAC$', Bob should be able to believe that SMP is authored by Alice or Bob,
but unable to prove that to any third party.

\subsubsection{Output Construction}

Alice constructs the binary output according to the this list:

\begin{enumerate}
	\item Magic header ``0x61390c02'', which means ``OWECDH'' in Base64. 4 bytes.
	\item A CBOR (RFC 8949) payload according to the following table.
\end{enumerate}

\begin{center}
	\begin{tabu}{lll}
		\toprule
		Field Name & Type          & Description                                          \\
		\midrule
		{ver}      & {uint (t0)}   & {Version indicator ``1''}                            \\
		{recvhint} & {uint (t0)}   & {Last 16 bits of ``to\_key'' as an unsigned integer} \ (hex)\
		{hmac}     & {buffer (t2)} & {$MAC$}                                              \\
		{iv}       & {buffer (t2)} & {$V$}                                                \\
		{ephem}    & {buffer (t2)} & {$Ephe.pub$}                                         \\
		{cblob}    & {buffer (t2)} & {$EB$}                                               \\
		\bottomrule
	\end{tabu}
\end{center}

Then a binary output payload (`output.bin') is created.

\subsection{Human-Friendly Wrapped Format}

Alice constructs the human-friendly wrapped format (`output.cism1') according to this list of steps:

\begin{enumerate}
	\item Magic header \texttt{-----BEGIN CISMAILv1 MESSAGE-----}.
	\item Base64 encoding of `output.bin' with line-wrapping.
	\item Magic footer \texttt{-----END CISMAILv1 MESSAGE-----}.
\end{enumerate}

Example:

\begin{lstlisting}
-----BEGIN CISMAILv1 MESSAGE-----
OWECDHBGSjkwEYuEbL+jEp8CFpQwV9OSJfm5Vs4yW+43BZb0FoYUvubjBk+F32lmsM/ik1
tGhyNtC5plFen+Jrn1XmcKZB6pRi0rCF7teFTAWAqSxw2XRdSYqzVO2V5puJgKMAqdN4Af
f7EgePgcXXCwjiEtvwvfLHeDn4PXslNRVAmbs7HyC0+2TXwWXOrP9gGmJakjzY/PMMPOG+
H40NmPzXXJBf2NIwJfjtSGnrw+Tdb0wuxLsIRlTyaPaRzOxQfwjxH8cUbHXX7y/8ygG7WA
uWVPdcZPyeisjic1qZyzTSI+yoo2JvzX1qZii/NlI49iUCh6EsA45YLoIVLGBse9CL8AXp
z+BgcYaCTae3qgahVNTIez3wWoT1wNIEr6ecJx9Z8/gcqX2wL/rowGEA2l1yOK+sOxQmM6
cS52rzkpMTiVdDaC4wwCjffxRnpwQ7Fgf5n+8uBmjFXGsQ7zla5eVyEXkLhjW8K9fZgW75
eOhB3gseubdMZwTp3F1RDWwgCbDbGH+TQCAb235cE6N+7Vf4Gsc2uZUU5rYeLptq0MW/sr
mTF8FpVMvKtAaOZScvlovqb86eM+W4q7DVNHVqN6EySAG4HoCQl6ZqrVAVW4HvOcI8B89E
pmaD7iGgtXYG3CrPgbc2yjldq8k1x61Hajn8tVzUvS39Ta+H0ZwWhy
-----END CISMAILv1 MESSAGE-----
\end{lstlisting}






\section{Embedded Dependencies}

The Cismail protocol requires the following protocols as dependencies:

\begin{enumerate}
	\item Message Payload Layer: Basic Mail Message Markup (BMMM)
	\item Channel Management Layer: One-Way ECDH-Hybrid Temporary Channel Protocol (cipher: AES-256-CFB)
\end{enumerate}

This document includes the draft specifications for the dependencies.
The formal versions of them shall have their own independent specification documents.

// TODO

% \subsection{Simple Person Keyquard Object}




\end{document}
