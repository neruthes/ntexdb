\documentclass[a4paper,11pt]{article}
\documentclass[a4paper,11pt]{article}
\usepackage{titling}
\usepackage{enumerate,enumitem,paralist}

% Use letterhead1 library
\input{/home/neruthes/DEV/ntexlibs/lib/letterhead1.tex}








% Fonts
\setmainfont{NewComputerModern}
\setromanfont{NewComputerModern}
\setsansfont{Inter}
\setmonofont{JetBrains Mono NL}
\setCJKmainfont{FandolSong}
\setCJKromanfont{FandolSong}
\setCJKsansfont{Noto Sans CJK SC}
\setCJKmonofont{Noto Sans CJK SC}

% Use alternative geometry
\newgeometry{textwidth=42em,tmargin=45mm,bmargin=33mm}

% % Paragraph style
% \setlength{\parindent}{2em}
% \setlength{\parskip}{4pt}

\newcommand{\Nmaketitle}[1]{%
    \begin{minipage}{\linewidth}%
        \parskip=6pt%
        \center%
        \rmfamily%
        \fontspec[Ligatures=TeX]{Latin Modern Roman}%
        {\LARGE#1}\par%
        {\sffamily\normalsize\theauthor}\par%
        {\sffamily\normalsize\thedate}%
    \end{minipage}\par
    \vspace{15pt}\hrule%
    \vspace{15pt}%
    \renewcommand{\ifootcornerL}[0]{#1}%
}
\newcommand{\Nmaketoc}[0]{\vspace{20pt}\hrule\tableofcontents\vspace{20pt}\hrule}


\renewcommand{\letterfooterfont}[0]{\sffamily\footnotesize}
\renewcommand{\ifootcornerL}[0]{A Note by \theauthor}
\renewcommand{\ifootcornerR}[0]{Page \thepage}

\newcommand{\icode}[1]{\texttt{\footnotesize#1}}


\fontsizepteleven
% \fontsizeptwelve
\docsetup{Autonomous Timeline Protocol}{Neruthes}{2025-05-06 (Manuscript WIP)}
\linespread{1.05}







\begin{document}

\fulldoctitle



\section*{Abstract}
The author attempted to make a static website generator that produces ActivityPub-compatible profile and feed
with the intention to allow federated subscription for users on Mastodon instances.
However, the attempt was unsuccessful for various causes.

As as result, the author hereby proposes an alternative approach to the new era of timeline-based social feed.
The protocol may be addressed as ``Autonomous Timeline Protocol'' (ATLP).


\section{Principles}
The protocol should...
\begin{enumerate}
    \item Be compatible with existing reader apps;
    \item Be easy to interact with from mobile clients;
    \item Be easy to publish content as static websites;
    \item Allow easy social spreading of feed items; and
    \item Be easy to migrate data from site to site.
\end{enumerate}



\section{Components}
The protocol consists of several components.

\begin{description}
    \item[Content Management Layer] Organize one's own content in a git repo like most static website generators.
    \item[Online Presence Layer] Cloudflare Pages, GitHub Pages, Vercel, etc.
        Includes human-friendly web frontend for browsers and formatted data for machines.
    \item[Client App Layer] A place to subscribe, relate, etc.
\end{description}


\subsection{Content Management Layer}
// TODO

\subsection{Online Presence Layer}
We want an extension of RSS. Namely, <link> tags with custom values on the ``rel'' attribute.

We call them subfeeds. Subfeeds are optional.
To have an ATLP main feed without any subfeed, you would effectively have a traditional RSS feed.

\begin{description}
    \item[rel=``retweets''] URL of dynamically generated list of recent retweets.
    \item[rel=``likes''] URL of dynamically generated list of recent likes.
    \item[rel=``inbox-public''] URL of received pings in the inbox.
\end{description}


\subsection{Client App Layer}

The generating and uploading of the dynamically generated data files may be left for specific client app implementations.
For example, a client app may allow the user to input a Cloudflare token so it pushes these alternative RSS files to an R2 object.


\section{Inbox Features}
Instance sites may optionally support inbox features besides the publishing.

\subsection{Comment Notification}
Like how pingback worked in ancient times,
the site manifest may declare inbox.notif=smtp and inbox.notifEndpoint=bot@example.com
to indicate an email address which accepts post-reply item URLs.

Behind the address, there may be a Cloudflare email worker who inserts to the optional inbox-public subfeed.







\end{document}



% https://pub-714f8d634e8f451d9f2fe91a4debfa23.r2.dev/ntexdb/fc8eee09aa4b118f50aa8ac7/note-20240625-oamarketing.tex
% https://pub-714f8d634e8f451d9f2fe91a4debfa23.r2.dev/ntexdb/ff8eb804647bf6b0a38b71f8/note-20240625-oamarketing.pdf
