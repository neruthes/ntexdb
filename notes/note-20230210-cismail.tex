\documentclass[a4paper,11pt]{article}
\usepackage{tocloft,tabu}

% Use letterhead1 library
\input{/home/neruthes/DEV/ntexlibs/lib/letterhead1.tex}

\usepackage{amssymb}

\usepackage{paralist,enumerate,enumitem}





% Fonts
\setmainfont{NewComputerModern}
\setromanfont{NewComputerModern}
\setsansfont{Inter}
\setmonofont{JetBrains Mono NL}
\setCJKmainfont{FandolSong}
\setCJKromanfont{FandolSong}
\setCJKsansfont{Noto Sans CJK SC}
\setCJKmonofont{Noto Sans CJK SC}

% Use alternative geometry
\renewcommand{\letterfooterwidth}[0]{44em}

\newcommand{\refreshgeometry}[0]{
    \newgeometry{a4paper,textwidth=\letterfooterwidth,tmargin=45mm,bmargin=29mm}
}
\newcommand{\fontsizepteleven}[0]{
    \renewcommand{\letterfooterwidth}[0]{40em}
    \refreshgeometry
}
\newcommand{\fontsizeptwelve}[0]{
    \renewcommand{\letterfooterwidth}[0]{37em}
    \refreshgeometry
}

% % Some dimensions
\setlength{\tabulinesep}{3pt}
% \setlength{\parindent}{2em}
% \setlength{\parskip}{4pt}

\newcommand{\simpledoctitle}[0]{%
    \begin{minipage}{\linewidth}%
        \parskip=6pt%
        \center%
        \rmfamily%
        \fontspec[Ligatures=TeX]{Latin Modern Roman}%
        {\LARGE\cacheddoctitlebig}\par\vspace{4pt}%
    \end{minipage}\par
    \vspace{15pt}\hrule%
    \vspace{15pt}%
    \renewcommand{\ifootcornerL}[0]{\cacheddoctitlefooter}%
}
\newcommand{\fulldoctitle}[0]{%
    \begin{minipage}{\linewidth}%
        \parskip=6pt%
        \center%
        \rmfamily%
        \fontspec[Ligatures=TeX]{Latin Modern Roman}%
        {\LARGE\cacheddoctitlebig}\par\vspace{4pt}%
        {\sffamily\normalsize\cacheddocauthor}\par%
        {\sffamily\normalsize\cacheddocdate}%
    \end{minipage}\par
    \vspace{15pt}\hrule%
    \vspace{15pt}%
    \renewcommand{\ifootcornerL}[0]{\cacheddoctitlefooter}%
}
\newcommand{\setrealdisplaytitle}[1]{
    \renewcommand{\cacheddoctitlebig}[0]{#1}
    \renewcommand{\cacheddoctitlefooter}[0]{#1}
}
\newcommand{\Nmaketoc}[0]{\vspace{20pt}\hrule\tableofcontents\vspace{20pt}\hrule}


\renewcommand{\letterfooterfont}[0]{\sffamily\footnotesize}
\renewcommand{\ifootcornerL}[0]{A Note by \theauthor}
\renewcommand{\ifootcornerR}[0]{Page \thepage}

\newcommand{\icode}[1]{\texttt{\footnotesize#1}}



\newcommand{\Asection}[1]{
    \section*{#1}
    \addcontentsline{toc}{section}{#1}
}
\newcommand{\Asubsection}[1]{
    \subsection*{#1}
    \addcontentsline{toc}{subsection}{#1}
}



\fontsizepteleven
% \fontsizeptwelve
\docsetup{Cismail Protocol Specification Version 1}{Neruthes}{2023-02-17 (WIP)}
\linespread{1.1}







\begin{document}
% \raggedright

\fulldoctitle


\section*{Abstract}

This document describes the design specification of the Cismail protocol.
``CIS'' in Cismail is short for cryptographically innovative-and-secure.

This is an attempt to put the 7-layer model%
\footnote{(WIP) 2023. Neruthes. Model of Seven Layers for Cryptographic Communication Protocols. [\href{https://pub-714f8d634e8f451d9f2fe91a4debfa23.r2.dev/keep/ntexdb/note-20230206-crypt-suite.pdf--6c7cbbc319948e0dd9e2ae6b5fcdfa27.pdf}{PDF link}]}
(of cryptographic communication protocols) in real use.

Cismail is a minimalist protocol which is designed to offer efficient cryptographic protection
with a workflow similar to existing practices with the OpenPGP protocol (RFC 4880).
It serves similar scenarios.

At this stage, the Cismail protocol is simple and straightforward.
It allows a plaintext message to be encrypted for a recipient from an author with deniable authentication.




\section{Sample Workflow}

\subsection{Background}

Alice wants to write to Bob.

They have exchanged their Cismail public keys (age) already before this story begins.

\subsection{Writing Message}

All Alice wants to say is a simple phrase consisting of 5 words: Lorem ipsum dolor sit amet.

Therefore, Alice has a UTF-8 plaintext raw message:

\begin{lstlisting}
Lorem ipsum dolor sit amet.
\end{lstlisting}

\subsection{Message Packaging Layer: Basic Mail Message Markup (BMMM)}

Alice needs to wrap the raw message into a mail payload `$MP$' by adding a header,
which consists of a few LF-delimited data fields and an ending mark (a line consisting of exactly 10 ``-'' characters).
These data fields allow the recipient to understand the context of the raw text.

\begin{lstlisting}
p: bmmm-1.0
app: cismail-format-v1
time: 1676524116
from: Alice
to: Bob
from_key: <Alice.kpid>
to_key: <Bob.kpid>
markup: plain
garbage_bytes: <Some random bytes ...>
----------

Lorem ipsum dolor sit amet.
\end{lstlisting}

The definitions of the data fields in the header will be discussed later.

Note: Random garbage bytes are inserted to make sure that the length (in bytes) of the output is always a multiple of 256.
If the raw input is longer than 4096, the length of the output is always a multiple of 1024.

\subsection{Session Management Layer: Null Session Protocol}

Null Session Protocol is a really simple Session Management Layer protocol.
It does nothing at all.
It maintains no state and adds no prefix.
It terminates after finishes echoing the first message going into it.

\subsection{Channel Management Layer: Universal Cipher Suite Plugging Wrapper}

The encryption operation is delegated to an appropriate program
which is capable of encrypting for a specific type of recipient identifier.
In this example, the type of recipient identifier is ``age''.

The program is used to calculate the encrypted blob `$EB$'.

\begin{displaymath}
	EB = Prog.enc(~Bob_{pub}~,~MP~)
\end{displaymath}

\subsection{Human-Friendly Wrapped Format}

Alice constructs the human-friendly wrapped format (`output.cism1') according to this list of steps:

\begin{enumerate}
	\item Magic header \texttt{-----BEGIN CISMAILv1 MESSAGE-----}.
	\item Base64 encoding of `output.bin' with line-wrapping.
	\item Magic footer \texttt{-----END CISMAILv1 MESSAGE-----}.
\end{enumerate}

Example:

\begin{lstlisting}
-----BEGIN CISMAILv1 MESSAGE-----
OWECDHBGSjkwEYuEbL+jEp8CFpQwV9OSJfm5Vs4yW+43BZb0FoYUvubjBk+F32lmsM/ik1
tGhyNtC5plFen+Jrn1XmcKZB6pRi0rCF7teFTAWAqSxw2XRdSYqzVO2V5puJgKMAqdN4Af
f7EgePgcXXCwjiEtvwvfLHeDn4PXslNRVAmbs7HyC0+2TXwWXOrP9gGmJakjzY/PMMPOG+
H40NmPzXXJBf2NIwJfjtSGnrw+Tdb0wuxLsIRlTyaPaRzOxQfwjxH8cUbHXX7y/8ygG7WA
uWVPdcZPyeisjic1qZyzTSI+yoo2JvzX1qZii/NlI49iUCh6EsA45YLoIVLGBse9CL8AXp
z+BgcYaCTae3qgahVNTIez3wWoT1wNIEr6ecJx9Z8/gcqX2wL/rowGEA2l1yOK+sOxQmM6
cS52rzkpMTiVdDaC4wwCjffxRnpwQ7Fgf5n+8uBmjFXGsQ7zla5eVyEXkLhjW8K9fZgW75
eOhB3gseubdMZwTp3F1RDWwgCbDbGH+TQCAb235cE6N+7Vf4Gsc2uZUU5rYeLptq0MW/sr
mTF8FpVMvKtAaOZScvlovqb86eM+W4q7DVNHVqN6EySAG4HoCQl6ZqrVAVW4HvOcI8B89E
pmaD7iGgtXYG3CrPgbc2yjldq8k1x61Hajn8tVzUvS39Ta+H0ZwWhy
-----END CISMAILv1 MESSAGE-----
\end{lstlisting}






\section{Embedded Dependencies}

The Cismail protocol requires the following protocols as dependencies:

\begin{enumerate}
	\item Message Payload Layer: Basic Mail Message Markup (BMMM)
	\item Channel Management Layer: Universal Cipher Suite Plugging Wrapper
\end{enumerate}

This document includes the draft specifications for the dependencies.
The formal versions of them shall have their own independent specification documents.

% // TODO

\subsection{Universal Cipher Suite Plugging Wrapper}

\subsubsection{Keypair Identifier}

A Keypair Identifier is a URI for a cryptographic identity.

A good short form for the term may be ``kpid''.

The string consists of 4 fields, as delimited by colon characters.

\begin{center}
	\begin{tabu}{ll}
		\toprule
		Field Name     & Description                         \\
		\midrule
		Static Header  & Static string ``kpid''.             \\
		Toolkit Name   & Consult the list of toolkits below. \\
		Key Identifier & The toolkit-specific identifier.    \\
		\bottomrule
	\end{tabu}
\end{center}

Examples:

\begin{itemize}
	\item kpid:age:age1s4zpwvrypemsn7ckn38uauedncy9m9yrn7dyak2trc7vst7mgpqskds8lg
\end{itemize}

List of toolkits:

\begin{description}
	\item[age] https://github.com/FiloSottile/age
\end{description}

\subsubsection{Encryption Operation}

\begin{displaymath}
	EB = Prog.enc(~K_{pub}~,~Msg~)
\end{displaymath}

\subsubsection{Decryption Operation}

\begin{displaymath}
	Msg = Prog.dec(~K_{pri}~,~EB~)
\end{displaymath}

\subsubsection{Wrapping}

\begin{description}
	\item[Bare Mode]
		Makes no wrapping at all.
		The application using this mode should have its own strategy to find out the correct toolkit.
	\item[Standard Mode]
		// TODO
\end{description}



\end{document}
