\documentclass[a4paper,11pt]{article}
\usepackage{tocloft,tabu}

% Use letterhead1 library
\input{/home/neruthes/DEV/ntexlibs/lib/letterhead1.tex}

\usepackage{amssymb}

\usepackage{paralist,enumerate,enumitem}





% Fonts
\setmainfont{NewComputerModern}
\setromanfont{NewComputerModern}
\setsansfont{Inter}
\setmonofont{JetBrains Mono NL}
\setCJKmainfont{FandolSong}
\setCJKromanfont{FandolSong}
\setCJKsansfont{Noto Sans CJK SC}
\setCJKmonofont{Noto Sans CJK SC}

% Use alternative geometry
\renewcommand{\letterfooterwidth}[0]{44em}

\newcommand{\refreshgeometry}[0]{
    \newgeometry{a4paper,textwidth=\letterfooterwidth,tmargin=45mm,bmargin=29mm}
}
\newcommand{\fontsizepteleven}[0]{
    \renewcommand{\letterfooterwidth}[0]{40em}
    \refreshgeometry
}
\newcommand{\fontsizeptwelve}[0]{
    \renewcommand{\letterfooterwidth}[0]{37em}
    \refreshgeometry
}

% % Some dimensions
\setlength{\tabulinesep}{3pt}
% \setlength{\parindent}{2em}
% \setlength{\parskip}{4pt}

\newcommand{\simpledoctitle}[0]{%
    \begin{minipage}{\linewidth}%
        \parskip=6pt%
        \center%
        \rmfamily%
        \fontspec[Ligatures=TeX]{Latin Modern Roman}%
        {\LARGE\cacheddoctitlebig}\par\vspace{4pt}%
    \end{minipage}\par
    \vspace{15pt}\hrule%
    \vspace{15pt}%
    \renewcommand{\ifootcornerL}[0]{\cacheddoctitlefooter}%
}
\newcommand{\fulldoctitle}[0]{%
    \begin{minipage}{\linewidth}%
        \parskip=6pt%
        \center%
        \rmfamily%
        \fontspec[Ligatures=TeX]{Latin Modern Roman}%
        {\LARGE\cacheddoctitlebig}\par\vspace{4pt}%
        {\sffamily\normalsize\cacheddocauthor}\par%
        {\sffamily\normalsize\cacheddocdate}%
    \end{minipage}\par
    \vspace{15pt}\hrule%
    \vspace{15pt}%
    \renewcommand{\ifootcornerL}[0]{\cacheddoctitlefooter}%
}
\newcommand{\setrealdisplaytitle}[1]{
    \renewcommand{\cacheddoctitlebig}[0]{#1}
    \renewcommand{\cacheddoctitlefooter}[0]{#1}
}
\newcommand{\Nmaketoc}[0]{\vspace{20pt}\hrule\tableofcontents\vspace{20pt}\hrule}


\renewcommand{\letterfooterfont}[0]{\sffamily\footnotesize}
\renewcommand{\ifootcornerL}[0]{A Note by \theauthor}
\renewcommand{\ifootcornerR}[0]{Page \thepage}

\newcommand{\icode}[1]{\texttt{\footnotesize#1}}



\newcommand{\Asection}[1]{
    \section*{#1}
    \addcontentsline{toc}{section}{#1}
}
\newcommand{\Asubsection}[1]{
    \subsection*{#1}
    \addcontentsline{toc}{subsection}{#1}
}



\fontsizepteleven
% \fontsizeptwelve
\docsetup{Music Streaming Service Protocol Architecture Design}{Neruthes \& Catten Linger}{2022-05-24 version 0.3.2}
\linespread{1}

% \FontsetLegal
\FontsetOldstyle
% \FontsetBook







\begin{document}

\fulldoctitle






\section*{Abstract}

This protocol is designed to enable a distributed network for music stream service,
where everyone may host their own music inventories and exchange access with each other.

The current version is a working draft. It is not a final resolution and does not reflect the final consensus of the authors.










\Nmaketoc\clearpage












\section{Roles}



\subsection{Inventory Server}

Responsibilities:

\begin{compactitem}
	\item Serves music files through HTTP/HTTPS.
	\item Maintains a self-declaration file for information of this inventory.
	\item Maintains an index for available music files with metadata (ID3 tags).
	\item (Optional) Extract cover images from music files in a dedicated cache directory.
\end{compactitem}



\subsection{Index Server}

Responsibilities:

\begin{compactitem}
	\item Maintains a list of upstream inventory servers.
	\item Periodically collects upstream catalogs to build an aggregated catalog.
	\item Disambiguation for name collisions among artists.
	\item Decides whether the tracks (in different albums) sharing the same artist name are really performed by the same artist.
	\item Uniquely identifies each album.
	\item (Optional) Implement Search APIs.
	\item (Optional) Extract cover images from upstream music files (or dedicated cache directories).
	\item (Optional) Selectively applies patches to track metadata, e.g. consolidate artist names.
	\item (Optional) Uses rules to mask certain tracks in an inventory server from being indexed.
\end{compactitem}

Notes:

\begin{compactitem}
	\item Each album should be identified by the Cartesian product of album artist and album title.
\end{compactitem}














\subsection{Player Server: Web Player}

Responsibilities:

\begin{compactitem}
	\item Follows index servers and provides an in-browser audio player for end users.
	\item Implement Search APIs.
\end{compactitem}

\subsection{Player Server: Independent App}

Responsibilities:

\begin{compactitem}
	\item Follows index servers and plays music files for end users.
\end{compactitem}

\subsection{Subscriber Client}

Responsibilities:

\begin{compactitem}
	\item Subscribes to a player server and plays music files for end users.
\end{compactitem}






















\section{APIs}


\subsection{Inventory Server}

\subsubsection{The MusicSite File}

This is a JSON file at \icode{\$siteurl\_main/.MusicSite.json}.

The file shall include the following properties:

\begin{tabu}{|l|X|}
	\hline
	{Property}      & {Description}                                                       \\
	\hline
	{title}         & {Human-readable title of this inventory.}                           \\
	{siteurl\_main} & {Canonical URL of this inventory.}                                  \\
	{siteurl}       & {Array of alternative URLs.}                                        \\
	{peers}         & {(Optional) Array of other inventories which the owner recommends.} \\
	\hline
\end{tabu}

Example:

\begin{lstlisting}
{
    "title": "Example Music Site",
    "siteurl_main": "https://music.example.com/music/",
    "siteurl": [
        "https://music.example.com/music/",
        "http://10.0.233.10:12345/disk2/audio/music/",
        "ftp://10.0.233.10/music/"
    ],
    "peers": [
        "https://anothermusicservice.com/"
    ],
    "features": {}
}
\end{lstlisting}

\subsubsection{The MusicCatalog File}

This is a JSON file at \icode{\$siteurl\_main/.MusicCatalog.json}. An array of tracks.

Each track shall include the following properties:

\begin{tabu}{|l|X|}
	\hline
	{Property} & {Description}                                                                  \\
	\hline
	{path}     & {Relative path from this catalog file.}                                        \\
	{size\_KB} & {Size of this music file in KB (1 KB = 1024 bytes).}                           \\
	{album}    & {Album name.}                                                                  \\
	{title}    & {Track title.}                                                                 \\
	{artist}   & {Track artist.}                                                                \\
	{disc}     & {(Nullable) Disc number if the album has multiple discs, starting from ``1''.} \\
	{track}    & {Sequential position of the track in the disc, starting from ``01''.}          \\
	\hline
\end{tabu}

Example:

\begin{lstlisting}
[
    {
        "path": "Genshin/01_风与异乡人/01 原神.flac",
        "size_KB": "19704",
        "album": "风与异乡人",
        "title": "原神",
        "artist": "陈致逸/HOYO-MiX",
        "disc": "",
        "track": "01"
    }
]
\end{lstlisting}






% \subsection{Index Server \& Player Server}


% \newcommand{\restfulapi}[3]{
% 	\fbox{
%         \begin{minipage}{0.9\linewidth}
%             \parskip=7pt
%             \vspace{5pt}
%             % $\blacksquare$ \textsf{#1}
%             $\blacksquare$ {#1}

%             % Endpoint: 
%             \icode{#2}

%             % \textbf{Note}:\hspace{1em}#3
%             Note:\hspace{1em}#3
%             \vspace{5pt}
%         \end{minipage}
%     }
% }


% \subsubsection{Entity APIs}

% % \subsubsection{Track}

% \restfulapi{Get Track Info}{GET /api/entity/track/[TrackID]}{None.}

% % \subsubsection{Album}

% \restfulapi{Get Album Info}{GET /api/entity/album/[AlbumID]}{None.}

% % \subsubsection{Artist}

% \restfulapi{Get Artist Info}{GET /api/entity/artist/[ArtistID]}{None.}

% \subsubsection{Search APIs}

% \restfulapi{Wild Search}{GET /api/search?query=[Criteria]}{Use this API to search in all fields}

% \restfulapi{Scoped Search}{GET /api/search?query=[Criteria]\&scopes=[Scope1,Scope2]}{Use this API to search in specified ranges, where valid scopes include `title', `album', and `artist'.}







\section{Optional Features}

Optional features are not mandatory. Implementations may decide whether they want to support any of these features.

\subsection{Ephemeral Inventory Access Model (EIAM)}

In certain circumstances, the Inventory Server would like to impose access control restrictions, making their hosting less public. Although Inventory Servers are supposed to be protected by CDNs, their maintainers may still have concerns on legal consequences for any potential copyright infringement.

With Ephemeral Inventory Access Model (EIAM), URLs are no longer stable. The Inventory may indicate its periodical token rotation so that its downstream parties may be able to redistribute the new token further down the stream.

EIAM is established on the old-school HTTP basic authentication protocol. For every new hour, the Inventory Server may regenerate a new token, which is known as `password' in the `HTTP basic authentication' context. The new token is calculated from two parts: the stable entropy string (ideally a UUIDv4 string) and the numerical representation of the current minute (\icode{\$date +\%Y\%m\%d.\%H\%M}). Join the two strings with a colon, then calculate the SHA-256 hash of the new string, then finally get its initial 30 characters.

Example code for regenerating ephemeral token:

% \begin{lstlisting}
% echo "$(cat .entropy-naspublicguest):$(date +%Y%m%d.%H%M)" \
%     | sha256sum | cut -c1-30
% \end{lstlisting}

If this feature is enabled, the Inventory Server shall indicate its policy by adding a new entry in the \icode{features.eiam} field of the MusicSite metadata file. The content object shall include the following properties:

\begin{tabu}{|l|l|X|}
	\hline
	{Property}        & {Type}   & {Description}                                                                                               \\
	\hline
	{\icode{minute}}  & {number} & {At which minute in every hour, the Inventory Server regenerates the new token. Nullable; defaults to `0'.} \\
	{\icode{guest}}   & {string} & {Which username the guests shall use with the token. Nullable; defaults to `guest'.}                        \\
	{\icode{entropy}} & {string} & {The entropy string.}                                                                                       \\
	\hline
\end{tabu}










\end{document}
