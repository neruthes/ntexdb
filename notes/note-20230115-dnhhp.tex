\documentclass[a4paper,11pt]{article}
\documentclass[a4paper,11pt]{article}
\usepackage{titling}
\usepackage{enumerate,enumitem,paralist}

% Use letterhead1 library
\input{/home/neruthes/DEV/ntexlibs/lib/letterhead1.tex}








% Fonts
\setmainfont{NewComputerModern}
\setromanfont{NewComputerModern}
\setsansfont{Inter}
\setmonofont{JetBrains Mono NL}
\setCJKmainfont{FandolSong}
\setCJKromanfont{FandolSong}
\setCJKsansfont{Noto Sans CJK SC}
\setCJKmonofont{Noto Sans CJK SC}

% Use alternative geometry
\newgeometry{textwidth=42em,tmargin=45mm,bmargin=33mm}

% % Paragraph style
% \setlength{\parindent}{2em}
% \setlength{\parskip}{4pt}

\newcommand{\Nmaketitle}[1]{%
    \begin{minipage}{\linewidth}%
        \parskip=6pt%
        \center%
        \rmfamily%
        \fontspec[Ligatures=TeX]{Latin Modern Roman}%
        {\LARGE#1}\par%
        {\sffamily\normalsize\theauthor}\par%
        {\sffamily\normalsize\thedate}%
    \end{minipage}\par
    \vspace{15pt}\hrule%
    \vspace{15pt}%
    \renewcommand{\ifootcornerL}[0]{#1}%
}
\newcommand{\Nmaketoc}[0]{\vspace{20pt}\hrule\tableofcontents\vspace{20pt}\hrule}


\renewcommand{\letterfooterfont}[0]{\sffamily\footnotesize}
\renewcommand{\ifootcornerL}[0]{A Note by \theauthor}
\renewcommand{\ifootcornerR}[0]{Page \thepage}

\newcommand{\icode}[1]{\texttt{\footnotesize#1}}


\fontsizepteleven
% \fontsizeptwelve
\docsetup{Domain Name Hyper Header Protocol (DNHHP) Specification Version 1.0 (Draft)}{Neruthes}{2023-01-15}
\linespread{1.1}







\begin{document}
% \raggedright

\fulldoctitle


\section*{Introduction}

Domain Name Hyper Header Protocol (DNHHP) is a lightweight extension to DNS TXT records.
DNHHP uses DNS TXT records as the transportation/representation layer to convey short messages in a key-value fashion.

The primary purpose of DNHHP is service discovery.
Such services can include SSH, HTTP, HTTPS, FTP, SMTP, IMAP, IMAP, POP3, etc.

\tableofcontents\clearpage




\section{A Straightforward Example}

Represented in DNS zone file format:

\begin{lstlisting}
@  IN  TXT  dnhhp webmaster-name John Appleseed
@  IN  TXT  dnhhp webmaster-emails johnappleseed@example.com
            webmaster@example.com
@  IN  TXT  dnhhp http-ports 80 2095 8000
@  IN  TXT  dnhhp https-ports 443 2096 8443
@  IN  TXT  dnhhp x-urlencoded-zeroconf IPv6%3Bbubulu%3B_ssh._tcp%3B\
            example.com%3B77.77.77.77%3Bfcc0%3A989c%3A7c13%3A9bf9%3A\
            32b7%3A%3A1%3B22%3B
\end{lstlisting}







\section{Design Synopsis}

DNHHP is built over DNS TXT records.

In the data part of a HTTP record, we have 3 sections delimited by whitespace:

\begin{enumerate}
	\item Protocol Version Indicator (PVI): Static text ``dnhhp''.
	\item Header Key Name (HKN): A short string to indicate the key of the record.
	\item Header Value Data (HVD): A short string to indicate the key of the record.
\end{enumerate}

\subsection{Protocol Version Indicator (PVI)}

The Protocol Version Indicator (PVI) should be static text ``dnhhp''.

In future, if DNHHP moves to a format incompatible with the current format, we will then suffix it with a version number.

\subsection{Header Key Name (HKN)}

The Header Key Name (HKN) is a short string to indicate the key of the record.
Although not technically necessarily,
it is advised to use only alphanumerical characters, hyphen, and underscore in the HKN,
to conform to the common practices in HTTP headers.

DNHHP is heavily inspired by HTTP headers.
And if possible, a HTTP server may include the DNHHP records of a domain name in the HTTP responses as headers
when it responds to HTTP requests for the domain name.
So domain name owners are advised to use HKNs which do not collide with the names of common HTTP headers.

\subsubsection{Reserved Key Names}

The prefix ``dnhhp-'' is a reserved namespace.
No actual record shall start with this prefix.

Some magics require the ``dnhhp-'' prefix to work properly:

\begin{enumerate}
	\item \textbf{dnhhp-urlencoded-}\\
	      When the value for a key (e.g. ``website-alt'') is too complicated, i.e. involving special characters which might cause syntax problems,
	      it may be preferable to process the value string with ``encodeURIComponent'' before writing into the TXT record.
	      In this case, instead of ``website-alt'', the domain owner writes ``dnhhp-urlencoded-website-alt'' as the HKN.
	      When a DNHHP lookup implementation finds that there is a ``dnhhp-urlencoded-'' prefixed record for the given query criteria,
	      it should return the decoded value of the encoded record, regardless of whether a non-prefixed plain record exists in the DNS answer.
	      A lookup implementation should allow querying prefixed keys from callers.
\end{enumerate}

\subsection{Header Value Data (HVD)}

It is allowed to include whitespace characters in the HVD section.
Implementations should handle this situation correctly.

For example:

\begin{lstlisting}
// In shell script
dnhhp_key="$(cut -d' ' -f2 <<< "$dnhhp_line")"
dnhhp_val="$(cut -d' ' -f3- <<< "$dnhhp_line")"

// In JavaScript
const dnhhpResultArr = dnhhp.lookupRaw('example.com', 'https-ports')[0];
const dnhhp_key = dnhhpResultArr[1];
const dnhhp_val = dnhhpResultArr.slice(2).join(' ');
\end{lstlisting}







\section{Standardized Lookup URI}

\subsection{Basic URI Format}

A lookup request may be represented by a URI in the following format:

\begin{lstlisting}
dnhhp://{{domain name}}/{{header key}}
\end{lstlisting}

For example:

\begin{lstlisting}
dnhhp://example.com/https-ports
\end{lstlisting}

\subsection{Extended URI Formats}

The following extended URI formats are recommended for lookup tool developers to implement,
but not a mandatory part of the URI standard.
A tool can simply ignore everything since the question mark and it should provide compatible results
by performing DNS queries over traditional port 53.

\begin{enumerate}
	\item \textbf{dnhhp://example.com/keyname?first}\\
	      Get the first record only.
	      Note that sorting of DNS records may be unreliable,
	      and this behavior might return different results across different DNS queries.
	\item \textbf{dnhhp://example.com/keyname?doh=1.1.1.1,dns.example.com}\\
	      Require the query to be made via DNS over HTTPS,
	      by giving a comma-delimited list of hostnames to request information from.
          Or leave empty ``?doh'' to let the lookup tool decide internally.
\end{enumerate}







\section{Configuration}

Take the ``https-ports'' for example, the domain owner should configure a TXT record like ``dnhhp https-ports 443 2096'',
in order to announce the available ports for HTTPS on the domain name.







\section{Possible Working Scenarios}

\subsection{Circumventing Port Blockage}

Sometimes, certain TCP/UDP ports, e.g. 80 and 443, of home cables may be blocked by ISPs.

When the home owner wants to maintain a public WWW/FTP/SMTP/IMAP server, the unavailability of default ports can cause troubles.
Theoretically, a visitor can add the port (which can be announced along with the domain name or the IP address) into the URL,
it is still a very unhappy experience.

Also, it is possible to create a ``gateway.example.com'' subdomain on Cloudflare (or similar services) with a redirect rule
which redirects all requests on ``http://gateway.example.com/'' toward ``http://actual.example.com:19980/'',
but a more beautiful solution may further help.

Imagine that a browser is smart enough to query a list of HTTPS ports by getting a ``https-ports'' DNHHP record via DNS
before it opens TCP or QUIC connections and tries another port after finding the previous one silent.






\section{Recommended Headers}

Here is a list of recommended headers with their definitions:

\begin{enumerate}
	\item \textbf{webmaster-name}\\
	      Contact information of the webmaster.
	\item \textbf{webmaster-emails}\\
	      A whitespace-delimited list of the email addresses of the webmaster.
	\item \textbf{https-ports} / \textbf{http-ports}\\
	      A whitespace-delimited list of ports which should be tried in the given order by WWW user agents.
	\item \textbf{website-alt}\\
	      A whitespace-delimited list of alternative locations of this website (e.g. a live mirror or a tarball/zip).
	      \\Should be urlencoded.
	\item \textbf{zeroconf}\\
	      The ``avahi-browse -atp'' compatible string (subtracting the initial 2 fields)
	      for zeroconf-compatible service discovery mechanisms.
	      \\Should be urlencoded.
\end{enumerate}







\section{Lookup Tools}

I have written a small shell script for DNHHP lookup purposes.

\href{https://github.com/neruthes/NDevShellRC/tree/master/bin/dnhhp}{https://github.com/neruthes/NDevShellRC/tree/master/bin/dnhhp}

Try retrieving ``https-ports'' and ``website-alt'' records on domain name ``neruthes.xyz'' with the tools provided above.

Run ``dnhhp all neruthes.xyz'' to get the complete list of the configured DNHHP headers.





\end{document}
