\documentclass[a4paper,12pt]{article}

\usepackage[a4paper,hmargin=23mm,tmargin=24mm,bmargin=22mm]{geometry}
\usepackage[dvipsnames]{xcolor}
\usepackage{calc}
\usepackage{amsmath,xltxtra,xunicode}
\usepackage{titlesec}
\usepackage{fontspec}
\usepackage{wasysym,oz,pxfonts,txfonts}
\usepackage{listings,paralist,enumitem}
\usepackage{tabu}
\usepackage{xeCJK}
\setCJKromanfont{Noto Serif CJK SC}


\usepackage[sfdefault]{inter}
\usepackage{courier}



\title{Comprehensive Liyue Pronunciation Guide}
\author{Neruthes}
\date{2022-01-14}


\setlength{\parindent}{0pt}
\setlength{\parskip}{7pt}
\setlength{\tabulinesep}{5pt}









\begin{document}
\rmfamily


\begin{minipage}{\linewidth}
    \center

    \fbox{\sffamily EARLY PREVIEW DRAFT}
    \vspace{15pt}

    {\scshape\sffamily\large Neoparia Akademia Konstantinou}

    \maketitle
\end{minipage}




\section*{Preface}

This guide is supposed to offer the readers a systemtic and comprehensive guide for pronouncing Liyue names of persons, locations,
and everything else.

We will go through comparisons with English, and eventually reach to a complete pronunciation table.


\section{Getting Started with Names of Persons}

Let's start with some persons.

\begin{tabu}{|l|X|}
    \hline
    {Name} & {Introduction} \\
    \hline
    {Hu Tao} & {This is a rare case when Chinese Pinyin and English pronounce similarly enough.} \\
    {Zhongli} & {Divided as ``Zhong Li''. Zhong: ``Zh'' is differs from ``Z'', like how ``Shell'' differs from ``Sell''. Li: Similar to European surname ``Lee''.} \\
    {Beidou} & {Divided as ``Bei Dou''. Bei: Similar to ``Ba'' in ``Baby''. Dou: ``D'' from ``Delta'', and ``Ou'' from ``Though''.} \\
    {Ganyu} & {Divided as ``Gan Yu''. Gan: Similar to ``Gun''. Yu: } \\
    \hline
\end{tabu}







\end{document}