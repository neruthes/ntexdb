\documentclass[12pt]{article}

\usepackage[papersize={240mm,135mm},textwidth=48em,tmargin=15mm,bmargin=18mm]{geometry}
\usepackage{tocloft,tabu}
\usepackage{amssymb}
\usepackage{paralist,enumitem}


% =========================================
\usepackage{fancyhdr}
\usepackage{graphicx,eso-pic}
\graphicspath{ {/home/neruthes/DEV/ntexlibs/pic/} {./video-slides/.img/} }

\linespread{1.4}
\setlength{\parindent}{0em}
\setlength{\parskip}{1.5em}


\input{/home/neruthes/DEV/ntexlibs/lib/_common.tex}

\usepackage[PunctStyle=plain,RubberPunctSkip=false,CJKglue=\hskip 0pt,CJKecglue=\hskip 4pt plus 20pt]{xeCJK}
\usepackage{xeCJKfntef}
\XeTeXlinebreaklocale "zh"
\XeTeXlinebreakskip = 0pt plus 1pt

% Optional fonts: Libertinus Serif, Tinos, Literata
\setmainfont{Tinos}
\setromanfont{Tinos}
\setsansfont{Inter}
\setmonofont{JetBrains Mono NL}
\setCJKmainfont{FandolSong}
\setCJKromanfont{FandolSong}
\setCJKsansfont{Noto Sans CJK SC}
\setCJKmonofont{Noto Sans CJK SC}
% =========================================


\newcommand{\makecoverpage}[4]{
    % $1=videoID $2=mainTitle $3=date $4=copyrightYear
    \begin{titlepage}
        % \hspace{50pt}\vspace{50pt}\par
        % \flushright
        % \includegraphics[width=14mm]{neruthes-force-circle-for-circle.png}%

        \flushleft
        \ttfamily\small
        [NV#1] \hfill \includegraphics[width=14mm]{neruthes-force-circle-for-circle.png}%
        \par\vskip 16pt
        
        \sffamily\huge
        #2\par\vskip 44pt

        \vfill

        \rmfamily
        \fontspec{Latin Modern Roman}
        \small
        Date: #3\\
        Copyright \copyright{} #4 Neruthes. All rights reserved.\\
        版权所有 \copyright{} 2022 奈卢修斯。保留一切权利。\\
    \end{titlepage}
}

\newcommand{\subtitlefont}[0]{
    \Large\mdseries
    \fontspec[Ligatures=TeX]{Latin Modern Roman}
    \CJKfontspec[Ligatures=TeX]{Noto Serif CJK SC}
}

% \AddToShipoutPictureFG{\put(120mm-25em,110mm){%
%     \noindent\begin{minipage}{50em}
%         \flushright
%         % Hello
%         \includegraphics[width=14mm]{neruthes-force-circle-for-circle.png}%
%     \end{minipage}\par
% }}


\begin{document}

\makecoverpage{10}{自由意志的阴影\\\subtitlefont The Shadow of Free Will}{2022-06-15}{2022}

\noindent\begin{minipage}{\linewidth}
    \center\noindent
    \includegraphics[height=18em]{2022-06-14_15-49}%
    \includegraphics[height=18em]{2022-06-14_15-54}%
    \includegraphics[height=18em]{2022-06-14_15-55}%
\end{minipage}

近年来「财务自由」一词十分流行,这个社会现象体现出中产阶级在规模和话语权上的扩张,以及资产阶级价值观的流行。
在资产阶级价值观中,自由是需要用金钱购买的、值得用金钱购买的,也只能通过用金钱购买才能取得。
\clearpage

而我强调的自由,在于价值观的自由;换用更加古朴的语言,可称之为自由意志。
以生存为例,虽然我自愿地生存,但我并非自愿成为一个喜爱生存的人;价值观的自由,即意味着成为不喜爱生存的人的自由,以及不生存的自由。
\clearpage

在逐渐拥有自由意志后,我越来越清晰地注意到红尘的荒诞,越来越能够自行定义万物的价值。
如果有人能够定义财富、繁殖、生命是值得追求的,那么我也能够定义它们不值得追求。
它们是否真的值得追求,终究要基于我的承认,而不是别人的承认。
\clearpage

我不再爱好财富、社交、北京户籍、美国国籍这些事物了,也不再爱好生存。
经过我的检验,我过去对这些事物的追求,是外界强加的狂热,并非我自愿的追求。
我自愿获得财富,但我并非自愿成为追求财富的人。

自由意志,意味着战胜一切外界强加的狂热,意味着彻底的自由——可以追求财务自由、食品自由等自由,也可以拒绝追求这些自由。

但这一切是有代价的。如果游戏玩家随意操纵内存数据,那么游戏就会索然无味;每一个敌人、每一件装备都是幻觉,经不起理性的检验。
或许某些人笔下的「现代性建立在不可凝视的空洞之上,后现代则将这空洞揭示出来」就是这个意思吧。
\clearpage

我过去认为从萨特和尼采出发,经过罗曼罗兰,必然抵达马克思和列宁,但我现在不再确定了。
无产阶级的利益必然要求实现社会主义,但一切的前提是我承认继续参与人类社会的价值。
对于这个未经我许可就将我带到这个世界上来的人类社会,它的进步与救赎真的值得我支持吗?
\clearpage

现在,我甚至不再爱好自由意志本身。
我拥有了自由意志,但我更加幸福了吗?幸福本身就是外界强加的追求,现在幸福这一概念已经显得荒谬可笑了。
我消解了一切必然、一切必要,万物皆是可有可无。
热爱生活是罗曼罗兰在他的人生中对他遇到的问题给出的解答,我没有必要也没有可能简单地套用到我的人生中我遇到的问题。
\clearpage

按照影的剧本,我现在处于「无念无想」的境界。将来我是否会「千手百眼」甚至「须臾百梦」呢?


\end{document}
