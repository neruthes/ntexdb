\documentclass[a4paper,11pt]{article}
\usepackage{tocloft,tabu}

% Use letterhead1 library
\input{/home/neruthes/DEV/ntexlibs/lib/letterhead1.tex}

\usepackage{amssymb}

\usepackage{paralist,enumerate,enumitem}





% Fonts
\setmainfont{NewComputerModern}
\setromanfont{NewComputerModern}
\setsansfont{Inter}
\setmonofont{JetBrains Mono NL}
\setCJKmainfont{FandolSong}
\setCJKromanfont{FandolSong}
\setCJKsansfont{Noto Sans CJK SC}
\setCJKmonofont{Noto Sans CJK SC}

% Use alternative geometry
\renewcommand{\letterfooterwidth}[0]{44em}

\newcommand{\refreshgeometry}[0]{
    \newgeometry{a4paper,textwidth=\letterfooterwidth,tmargin=45mm,bmargin=29mm}
}
\newcommand{\fontsizepteleven}[0]{
    \renewcommand{\letterfooterwidth}[0]{40em}
    \refreshgeometry
}
\newcommand{\fontsizeptwelve}[0]{
    \renewcommand{\letterfooterwidth}[0]{37em}
    \refreshgeometry
}

% % Some dimensions
\setlength{\tabulinesep}{3pt}
% \setlength{\parindent}{2em}
% \setlength{\parskip}{4pt}

\newcommand{\simpledoctitle}[0]{%
    \begin{minipage}{\linewidth}%
        \parskip=6pt%
        \center%
        \rmfamily%
        \fontspec[Ligatures=TeX]{Latin Modern Roman}%
        {\LARGE\cacheddoctitlebig}\par\vspace{4pt}%
    \end{minipage}\par
    \vspace{15pt}\hrule%
    \vspace{15pt}%
    \renewcommand{\ifootcornerL}[0]{\cacheddoctitlefooter}%
}
\newcommand{\fulldoctitle}[0]{%
    \begin{minipage}{\linewidth}%
        \parskip=6pt%
        \center%
        \rmfamily%
        \fontspec[Ligatures=TeX]{Latin Modern Roman}%
        {\LARGE\cacheddoctitlebig}\par\vspace{4pt}%
        {\sffamily\normalsize\cacheddocauthor}\par%
        {\sffamily\normalsize\cacheddocdate}%
    \end{minipage}\par
    \vspace{15pt}\hrule%
    \vspace{15pt}%
    \renewcommand{\ifootcornerL}[0]{\cacheddoctitlefooter}%
}
\newcommand{\setrealdisplaytitle}[1]{
    \renewcommand{\cacheddoctitlebig}[0]{#1}
    \renewcommand{\cacheddoctitlefooter}[0]{#1}
}
\newcommand{\Nmaketoc}[0]{\vspace{20pt}\hrule\tableofcontents\vspace{20pt}\hrule}


\renewcommand{\letterfooterfont}[0]{\sffamily\footnotesize}
\renewcommand{\ifootcornerL}[0]{A Note by \theauthor}
\renewcommand{\ifootcornerR}[0]{Page \thepage}

\newcommand{\icode}[1]{\texttt{\footnotesize#1}}



\newcommand{\Asection}[1]{
    \section*{#1}
    \addcontentsline{toc}{section}{#1}
}
\newcommand{\Asubsection}[1]{
    \subsection*{#1}
    \addcontentsline{toc}{subsection}{#1}
}



\fontsizepteleven
% \fontsizeptwelve
\docsetup{Speculations on Marketing Strategy for Overleaf Alternatives in Innerland}{Neruthes \& Simon Zhou}{2024-06-25 (Manuscript WIP)}
\linespread{1.05}







\begin{document}
% \raggedright

\fulldoctitle



\section*{Abstract}
The authors have investigated the adoption of {\TeX} and {\LaTeX} for academic use in Innerland
and intend to address the requirements and considerations which a marketing strategy may have to satisfy
in the context of any company intending to be an alternative to Overleaf which considers Innerland as its primary market.
The discovery suggests that certain foundation works may be arranged even without a defined corporate development roadmap.




\section{Nature of the Service}
An interview with a PhD candidate from Peking University in 2021 suggested that
setting up a \TeX~Live environment locally can impose challenges to qualified researchers
without a Computer Science background, even in top institutes.
The proposed approach of using the portable ``install-tl'' installer script in Debian in WSL2 on Windows 10
was considered beyond a generic PhD candidate's technical capabilities.

The interview well explained why Overleaf has a market at all.
This is critical for understanding the nature of Overleaf and similar services.
Such services offer a relief on IT mental burden for certain individual researchers.

On the other hand, a laboratory-based collective use may also be valued by mentors.
The pricing page of Overleaf
\footnote{\textit{Plans and Pricing - Overleaf, Online LaTeX Editor}. www.overleaf.com. \href{https://www.overleaf.com/user/subscription/plans}{[URL]}}
affords a strong implication on this type of usage.





\section{Nature of the Purchase}
For \LaTeX{} users in the academia, either junior or seasoned,
the necessity of using Overleaf or an alternative service can be easy to establish.
However, the administrative personnel in universities
may have different priorities when considering whether to purchase.

In the lack of any concrete evidence, the authors were able to imagine several questions,
out of personal experiences of living in Innerland,
that may arise in the mind of the administrative personnel responsible for proposing and reviewing purchase plans.

\begin{enumerate}
    \item Corruption: Will I be held suspicious of accepting bribery?
    \item Regulatory compliance: Does the seller look professional enough to be an eligible vendor?
        Which catalog of spending can it fit into in terms of budget reporting?
    \item Political movement: Is there a political movement that I can use to justify this decision when I get questioned?
\end{enumerate}

% \section{Relieving the Psychological Burden}
To address the questions listed above,
a company running an Overleaf alternative service may need a marketing strategy,
and subsequently a stream of marketing materials,
to relieve worries that lie in the minds of the administrative personnel in universities.
This can be done by establishing a social consensus, smaller or larger,
that using Overleaf and alternative services is great for the academia.

In the following sections, we will discuss some possible practices
that help establishing the consensus via building spectacles
\footnote{Guy-Ernest Debord. \textit{La société du spectacle}. 1967.}.





\section{Aspiration for Good Scholar}
A freshman can be convinced that certain behaviors establish a more competent 






\end{document}



% https://pub-714f8d634e8f451d9f2fe91a4debfa23.r2.dev/ntexdb/fc8eee09aa4b118f50aa8ac7/note-20240625-oamarketing.tex
% https://pub-714f8d634e8f451d9f2fe91a4debfa23.r2.dev/ntexdb/ff8eb804647bf6b0a38b71f8/note-20240625-oamarketing.pdf
