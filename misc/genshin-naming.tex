\documentclass[a4paper,11pt]{report}

\title{通用命名学}
\author{Neruthes}
\date{\today}

\usepackage[a4paper,textwidth=38em,vmargin=30mm]{geometry}
\usepackage[dvipsnames]{xcolor}
\usepackage{calc}
\usepackage{amsmath,xltxtra,xunicode}
\usepackage{titlesec}
\usepackage{fontspec}
\usepackage{wasysym,oz,pxfonts,txfonts}
\usepackage{longtable,tabu,listing,tocloft}


\usepackage[PunctStyle=plain,RubberPunctSkip=false]{xeCJK}
\XeTeXlinebreaklocale "zh"
\XeTeXlinebreakskip = 0pt plus 1pt

\setCJKmainfont{Noto Sans CJK SC}
\setCJKromanfont{Noto Serif CJK SC}
\setCJKsansfont{Noto Sans CJK SC}
\setCJKmonofont{Noto Sans CJK SC}


\begin{document}
\tabulinesep=6pt
\parindent=0pt
\parskip=5pt

\begin{titlepage}
	\maketitle
\end{titlepage}

\setcounter{tocdepth}{1}
\tableofcontents

\chapter*{前言}


\chapter{配队命名}

\section{固定搭配命名法}

部分固定搭配已经众所周知,所以我们承认并沿用这些固定搭配。

\begin{longtabu}{|c|c|X[c]|}
	\hline
	{固定搭配名称} & {简称} & {内容} \\
	\hline
	\endhead
	{国家队} & {-} & {班尼特 + 香菱 + 行秋 + 重云} \\
	{雷神国家队} & {雷国} & {班尼特 + 香菱 + 行秋 + 雷电将军} \\
	{万达国际} & {-} & {班尼特 + 香菱 + 达达利亚 + 枫原万叶} \\
	\hline
\end{longtabu}

\section{角色缩写命名法}

灵活搭配的阵容命名,需要组合阵容中的角色的单字简称。在条件允许的情况下,应当优先选用首选单字简称。但是,为特定修辞效果,允许使用备用单字简称。除非极端特殊需求,不应选用另外的单字简称。

\begin{longtabu}{|X[c]|c|c|}
	\hline
	{角色} & {首选单字简称} & {备用单字简称} \\
	\hline
	\endhead
	\hline
	\endfoot
	{迪卢克} & {迪} & {卢} \\
	{琴} & {琴} & {-} \\
	{莫娜} & {莫} & {-} \\
	{温迪} & {温} & {-} \\
	{可莉} & {可} & {-} \\
	{阿贝多} & {阿} & {-} \\
	{优菈} & {优} & {-} \\
	\hline
	{安柏} & {安} & {-} \\
	{丽莎} & {丽} & {-} \\
	{凯亚} & {凯} & {-} \\
	{班尼特} & {班} & {点} \\
	{罗莎莉亚} & {罗} & {修} \\
	\hline
	{刻晴} & {刻} & {-} \\
	{七七} & {七} & {-} \\
	{钟离} & {钟} & {-} \\
	{甘雨} & {甘} & {-} \\
	{魈} & {魈} & {-} \\
	{胡桃} & {胡} & {-} \\
	{申鹤} & {申} & {鹤} \\
	% {XX} & {XX} & {-} \\
	\hline
	{香菱} & {香} & {-} \\
	{行秋} & {行} & {秋} \\
	{凝光} & {凝} & {-} \\
	{北斗} & {北} & {-} \\
	{烟绯} & {烟} & {-} \\
	{云堇} & {云} & {-} \\
	{XX} & {XX} & {-} \\
	\hline
	{枫原万叶} & {万} & {-} \\
	{神里绫华} & {神} & {-} \\
	{宵宫} & {宵} & {-} \\
	{雷电将军} & {雷} & {-} \\
	{珊瑚宫心海} & {心} & {-} \\
	{荒泷一斗} & {荒} & {斗} \\
	\hline
	{八重神子} & {八} & {-} \\
	{九条裟罗} & {九} & {裟} \\
	{五郎} & {五} & {狗} \\
	\hline
	{达达利亚} & {达} & {公} \\
\end{longtabu}




\chapter{武器命名}



\chapter{圣遗物命名}

\section{圣遗物套装命名}

\begin{longtabu}{|c|X[c]|c|c|}
	\hline
	{产地} & {套装名称} & {主要简称} & {备用简称} \\
	\hline
	\endhead
	\hline
	\endfoot
	{蒙德} & {苍翠猎人} & {风} & {-} \\
	{} & {少女} & {少女} & {-} \\
	\hline
	{璃月} & {魔女} & {魔女} & {火} \\
	{} & {逆飞的流星} & {逆飞} & {-} \\
	\hline
	{稻妻} & {绝缘之旗印} & {绝缘} & {充能} \\
	{} & {追忆之注连} & {追忆} & {泄能} \\
\end{longtabu}




\chapter{密境命名}

\begin{longtabu}{|X[c]|c|l|c|c|}
	\hline
	{密境} & {地区} & {描述} & {主要名称} & {备用名称} \\
	\hline
	\endhead
	\hline
	\endfoot
	{铭记之谷} & {蒙德} & {圣遗物:风,少女} & {风本} & {-} \\
	{仲夏之庭} & {蒙德} & {圣遗物:如雷、平雷} & {雷本} & {-} \\
	{芬德尼尔之顶} & {蒙德} & {圣遗物:水、冰} & {冰本} & {-} \\
	{无妄引咎之宫} & {璃月} & {圣遗物:魔女、渡火} & {火本} & {-} \\
	{华池岩岫} & {璃月} & {圣遗物:宗室、骑士} & {宗室本} & {-} \\
	{山脊守望} & {璃月} & {圣遗物:千岩、苍白} & {千岩本} & {苍白本} \\
\end{longtabu}


\end{document}
