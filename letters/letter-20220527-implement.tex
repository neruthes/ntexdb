%!TEX TS-program = xelatex
%!TEX encoding = UTF-8 Unicode

\documentclass[a4paper,11pt]{article}
\input{/home/neruthes/DEV/ntexlibs/lib/letterhead1.tex}
\usepackage{datetime2}
\FontsetPaper
\linespread{1.2}


\begin{document}
\rmfamily\frenchspacing
\chineseindenting

\noindent
亲爱的安老师:
\vspace{1em}

见信好。本次来信,我想要介绍,围绕早前谈及的「世界的可实现性」之问题,我的后续思考。
若将来增添篇幅著书立说,《造物主的自我修养》当属首选标题。





\lettersection{一・实现的实现}

首先我们要检验上述「实现」之概念如何从人类这一物种的意识中被实现出来。
卡尔马克思早已论断,人类的历史是劳动与阶级斗争的历史。卡尔荣格也曾谈及,人对他人的认知不可避免地从自己的主观性出发。

劳动是人类演化史的主旋律。劳动将人类的先祖塑造成人类。
劳动是这样一种活动:某种主体出于自己的主观意识,向某种原料施加某种劳动,以谋求得到某种产出。

纵使水果与猎物可以当作自然的馈赠,将这些馈赠收入囊中的过程依然是劳动。
数千年前,尚且年轻的人类文明不可避免地套用了「劳动与产出」的经验,将世界上的物质普遍地诠释为某种劳动的产出。
于是,造物主的概念被人类创造了出来——既然我手中的美酒与长矛是劳动成果,那么这个世界本身也一定是某种「伟大的存在」的劳动成果吧。

被这样的演化史塑造的人类,难以抵御「劳动与产出」这副有色眼镜的诱惑。摘除主观性的出发点是不现实的,就像晶状体是眼球不可分割的组成部分。
但是,至少我们可以认识到,我们认识世界的方式以怎样的方式受到我们的主观性的限制。

我们如今仍然要沿用「劳动与产出」的范式吗?







\lettersection{二・质料与形式}

柏拉图与亚里士多德的争论言犹在耳,《雅典学院》之画作挂在每个人的灵魂里。
在这个时代,我们十分熟悉分子和原子,波函数和凝聚态也开始成为常识。但这似乎让我们更加淡忘,什么让一个水杯是水杯。

一个玻璃水杯,其质料与沙子别无二致。英特尔与台积电,也未能走得太远。
某个物品之所以合理地成为水杯,其本质在于,作者与读者的主观性达成了共识。
作者心怀「制作水杯」之目标,以沙子为原料,施加劳动;读者心怀对「水杯」的需求,审视这个劳动成果。
那么,在它被用于盛装液体之前,在它尚且停留在货架上时,它就已经充分地成为了水杯。

是否可以固执地相信沙子与水杯都只是二氧化硅?
是否可以固执地相信水杯与芯片都只是硅原子?
是否可以固执地相信芯片与铜板都只是强子?
是否可以固执地相信质子与中子都只是夸克?

若仅着眼于质料而无视形式,似乎必然走入正确的废话。
若上述的主观的固执是可以接受的,那么对正确的废话的主观的排斥同样是可以接受的。

若我们仍然试图解释物体与物体的区别、物质与物质的区别,那么我们必然需要接受,物质并非仅是质料,物质也是形式。

自然数集有无穷个成员,却只有空集这一质料,其他元素只是这一质料的不同形式。
无数种宏观物质可以还原成一百多种原子的不同形式,但原子也只是三种强子的不同形式。
若标准模型内每一种粒子都被证明为超弦这唯一的质料的不同形式,那么物质的形式要素就会比质料要素更加重要——
所有的物质归根结底都由相同的质料构成,则区分不同物质的唯一理由是它们的形式差异。

所以,我们甚至可以更加激进地认为,形式不仅是物质的首要要素,更是物质的唯一要素。







\lettersection{三・存在与创造}

先前安老师为「实现」一词给出的定义「从不存在变为存在的过程」十分恰当,值得沿用。

一些二氧化硅原子,从沙子变成水杯的过程,凝结了无差别人类劳动于其中。
人类在制造这个水杯时,究竟实现了什么?
人类以焓为手段、以熵为代价,将一坨二氧化硅变成了另一坨二氧化硅——沙子消失了,水杯被实现了。

这个水杯的质料是人造的吗?不是。硅原子和氧原子都是既有的。

这个水杯的形式是人造的吗?是,但不全是。从夸克到二氧化硅分子的部分是既有的,从二氧化硅分子到水杯的部分是人造的。

若我们激进地无视物质的质料要素,仅承认形式要素,那么我们会得到有趣的巧合——物质的人造性问题,与增值税问题十分相似。

增值税的征税对象是商品的增值过程;至于原料本身,则被认为是无价值的。
水果是果汁的原料,果汁厂需要从上游(果农)采购水果来制造果汁卖给下游(超市)。
水果价格与果汁价格之间的差异是增值部分,政府需要向果汁厂征收增值税。
但是,果农的上游是果树,上游采购费用为零。在增值税的意义上,水果是果农从虚空中掏出来的。

现代社会的高科技产品,远比水杯更加精密复杂。
即使芯片是台积电造的、屏幕是京东方造的、摄像头是索尼造的,一台手机仍然称得上是富士康造的。








\lettersection{四・世界之存在}

一切存在都必然是被某种主体从不存在实现出来的吗?
这个世界是被实现出来的吗?

我的解答已经蕴含在本文的起点。
我们只是太习惯于套用人类通过劳动改造世界的经验来诠释人类开始劳动之前的世界为何处于那种状态。
我们只是太习惯于将「劳动与产出」的范式投射给虚构的造物主。

或许本文未能正面回应原初的问题,但我已经尽力了;劝你不要不识好歹,手动滑稽。












\lettersignature{19em}{此致\par 奈卢修斯\par 二零二二年零五月二七日(金)}
\end{document}
