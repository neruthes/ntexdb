\input{/home/neruthes/DEV/ntexlibs/lib/paper-academic.tex}
\uselangzh

\usepackage{longtable}
\usepackage{tabu}

\setlength{\tabulinesep}{6pt}


\begin{document}
\imaketitle{通用命名学}{Neruthes}{[Early Preview Draft]}

\section*{简介}

本文旨在介绍一种通用的遣词造句方案,方便玩家之间交流。

\maketoc{}{}

\vspace{2em}
\hrule

\section{角色配装命名}

\subsection{示例}

\begin{itemize}
	\item 钟离 4 命,决斗精 4,绝缘 4,生岩暴伤,310。
	\item 90 枫原万叶 0 命,铁蜂刺精 1,风 4,精精精,999。
	\item 85 魈,和璞鸢精 1,风 2 散件,攻风暴伤,10210。
	\item 菲谢尔 1 命,85 阿莫斯精 1,苍白+骑士,攻物暴,10/11/1。
\end{itemize}

\subsection{组成结构}

\begin{tabu} {|l|X|}
	\hline
	{组件}         & {解释说明}                                                                                                                                                                                                          \\
	\hline
	{角色基本信息} & {写出名字、等级、命之座层数。可以省略等级,默认 90 级。可以省略命之座,默认 0 命。}                                                                                                                                 \\
	{武器}         & {写出武器名字、精炼层数。可以省略等级,默认 90 级。可以省略精炼,五星武器默认精 1,四星武器默认精 5。}                                                                                                              \\
	{圣遗物套装}   & {四件套可以省略数字。两件套配散件可以省略散件。两个两件套需要用加号连接。五散件可以省略套装信息。}                                                                                                                  \\
	{圣遗物主属性} & {写下后三件的主属性,均用主属性名字的第一个字。但有例外:元素精通简称「精」,暴击伤害简称「暴伤」。}                                                                                                                             \\
	{天赋等级}     & {用连续的数字写下角色天赋等级,包含命之座提供的等级,但不包含特殊因素(例如达达利亚提供的等级)。101010 可以简写为 310。存在超出 10 级的,可以写成 9119 这样的形式。9111 这样存在歧义的,写作 9/1/11 或 9/11/1。} \\
	\hline
\end{tabu}


\section{配队命名}

\subsection{固定搭配命名法}

部分固定搭配已经众所周知,所以我们承认并沿用这些固定搭配。


\begin{longtabu}{|c|c|X[c]|}
	\hline
	{固定搭配名称} & {简称} & {成员} \\
	\hline
	\endhead
	{国家队} & {-} & {班尼特 + 香菱 + 行秋 + 重云} \\
	{雷神国家队} & {雷国} & {班尼特 + 香菱 + 行秋 + 雷电将军} \\
	{万达国际} & {-} & {班尼特 + 香菱 + 达达利亚 + 枫原万叶} \\
	\hline
\end{longtabu}

成员列表排序规则:首先按照元素(火、水、风、雷、草、冰、岩的顺序)分组,然后每个组内按照体型(成男、少男、成女、少女、幼女的顺序)分组,然后每个组内按照所属国家(其他、蒙德、璃月、稻妻、须弥、枫丹、纳塔、至冬的顺序)分组,然后每个组内按照实装版本排序,最后每个组内按照稀有度(五星、四星)排序。


\subsection{角色缩写命名法}

灵活搭配的阵容命名,需要组合阵容中的角色的单字简称。在条件允许的情况下,应当优先选用首选单字简称。但是,为特定修辞效果,允许使用备用单字简称。除非极端特殊需求,不应选用另外的单字简称。

\begin{longtabu}{|X[2c]|X[c]|X[c]|}
	\hline
	{角色} & {首选单字简称} & {备用单字简称} \\
	\hline
	\endhead
	\hline
	\endfoot
	{迪卢克} & {迪} & {卢} \\
	{琴} & {琴} & {-} \\
	{莫娜} & {莫} & {-} \\
	{温迪} & {温} & {-} \\
	{可莉} & {可} & {-} \\
	{阿贝多} & {阿} & {-} \\
	{优菈} & {优} & {-} \\
	\hline
	{安柏} & {安} & {-} \\
	{丽莎} & {丽} & {-} \\
	{凯亚} & {凯} & {-} \\
	{班尼特} & {班} & {点} \\
	{芭芭拉} & {芭} & {} \\
	{雷泽} & {泽} & {} \\
	{诺艾尔} & {诺} & {} \\
	{砂糖} & {砂} & {} \\
	{迪奥娜} & {娜} & {猫} \\
	{菲谢尔} & {菲} & {皇} \\
	{罗莎莉亚} & {罗} & {修} \\
	\hline
	{刻晴} & {刻} & {-} \\
	{七七} & {七} & {-} \\
	{钟离} & {钟} & {-} \\
	{甘雨} & {甘} & {-} \\
	{魈} & {魈} & {-} \\
	{胡桃} & {胡} & {-} \\
	{申鹤} & {申} & {鹤} \\
	{白术} & {白} & {} \\
	\hline
	{香菱} & {香} & {-} \\
	{行秋} & {行} & {秋} \\
	{凝光} & {凝} & {-} \\
	{北斗} & {北} & {-} \\
	{烟绯} & {烟} & {-} \\
	{云堇} & {云} & {-} \\
	{重云} & {重} & {-} \\
	{辛焱} & {辛} & {-} \\
	\hline
	{枫原万叶} & {万} & {-} \\
	{神里绫华} & {神} & {-} \\
	{宵宫} & {宵} & {-} \\
	{雷电将军} & {雷} & {军} \\
	{珊瑚宫心海} & {心} & {-} \\
	{荒泷一斗} & {荒} & {斗} \\
	{八重神子} & {八} & {-} \\
	{神里绫人} & {人} & {舅} \\
	% {XX} & {XX} & {-} \\
	\hline
	{托马} & {托} & {马} \\
	{五郎} & {五} & {狗} \\
	{九条裟罗} & {九} & {裟} \\
	{早柚} & {早} & {柚} \\
	\hline
	{达达利亚} & {达} & {公} \\
\end{longtabu}




\section{武器命名}



\section{圣遗物命名}

\subsection{圣遗物套装命名}

\begin{longtabu}{|c|X[c]|c|c|}
	\hline
	{产地} & {套装名称} & {主要简称} & {备用简称} \\
	\hline
	\endhead
	\hline
	\endfoot
	{} & {角斗} & {角斗} & {-} \\
	{} & {流浪大地的乐团} & {乐团} & {-} \\
	\hline
	{蒙德} & {苍翠猎人} & {风} & {-} \\
	{} & {少女} & {少女} & {-} \\
	{} & {如雷的盛怒} & {如雷} & {雷} \\
	{} & {平息雷鸣的尊者} & {平雷} & {-} \\
	{} & {水} & {水} & {-} \\
	{} & {冰} & {冰} & {-} \\
	\hline
	{璃月} & {魔女} & {魔女} & {火} \\
	{} & {渡过烈火的贤人} & {渡火} & {-} \\
	{} & {磐岩} & {磐岩} & {岩} \\
	{} & {逆飞的流星} & {逆飞} & {-} \\
	{} & {昔日的宗室之仪} & {宗室} & {-} \\
	{} & {染血的骑士} & {骑士} & {染血} \\
	{} & {千岩} & {千岩} & {} \\
	{} & {苍白之火} & {苍白} & {} \\
	\hline
	{稻妻} & {绝缘之旗印} & {绝缘} & {充能} \\
	{} & {追忆之注连} & {追忆} & {泄能} \\
	{} & {华馆} & {华馆} & {防御} \\
	{} & {海染} & {海染} & {蚌埠/毒奶} \\
\end{longtabu}




\section{密境命名}

\begin{longtabu}{|c|X[c]|l|c|c|}
	\hline
	{地区} & {密境} & {描述} & {主要名称} & {备用名称} \\
	\hline
	\endhead
	\hline
	\endfoot
	{蒙德} & {铭记之谷} & {圣遗物:风,少女} & {风本} & {-} \\
	{ } & {仲夏之庭} & {圣遗物:如雷、平雷} & {雷本} & {-} \\
	{ } & {芬德尼尔之顶} & {圣遗物:水、冰} & {冰本} & {-} \\
	\hline
	{璃月} & {无妄引咎之宫} & {圣遗物:魔女、渡火} & {火本} & {-} \\
	{ } & {华池岩岫} & {圣遗物:宗室、骑士} & {宗室本} & {-} \\
	{ } & {岩} & {圣遗物:磐岩、逆飞} & {岩本} & {} \\
	{ } & {山脊守望} & {圣遗物:千岩、苍白} & {千岩本} & {苍白本} \\
	\hline
	{稻妻} & {} & {圣遗物:绝缘、追忆} & {绝缘本} & {充能本} \\
	{ } & {} & {圣遗物:宗室、骑士} & {防御本} & {-} \\
\end{longtabu}



\end{document}
