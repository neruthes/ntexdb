\documentclass[a4paper,11pt]{article}

\usepackage[a4paper,hmargin=23mm,tmargin=24mm,bmargin=22mm]{geometry}
\usepackage[dvipsnames]{xcolor}
\usepackage{calc}
\usepackage{amsmath,xltxtra,xunicode}
\usepackage{titlesec}
\usepackage{fontspec}
\usepackage{wasysym,oz,pxfonts,txfonts}
\usepackage{listings,paralist,enumitem}
\usepackage{tabu}


% Fonts
\usepackage[sfdefault]{inter}
% \usepackage{courier}
\usepackage{xeCJK}
\setmonofont{JetBrains Mono NL}
\setCJKromanfont{Noto Serif CJK SC}
\setCJKsansfont{Noto Sans CJK SC}
\setCJKmonofont{Noto Sans CJK SC}



% Metadata
\title{Wishnet: Reservse Web Protocol}
\author{Neruthes}
\date{2022-02-05 (version 0.1.0-pre1)}


\setlength{\parindent}{0pt}
\setlength{\parskip}{7pt}
\setlength{\tabulinesep}{5pt}







\newcommand{\inlinecode}[1]{
    {\fbox{\small\ttfamily{#1}}}
}

% Code blocks
\definecolor{codegreen}{rgb}{0,0.6,0}
\definecolor{codegray}{rgb}{0.5,0.5,0.5}
\definecolor{codepurple}{rgb}{0.58,0,0.82}
\definecolor{backcolour}{rgb}{0.95,0.95,0.95}
\lstdefinestyle{mystyle}{
    backgroundcolor=\color{backcolour},
    commentstyle=\color{codegreen},
    keywordstyle=\color{magenta},
    numberstyle=\tiny\color{codegray},
    stringstyle=\color{codepurple},
    basicstyle=\ttfamily\footnotesize,
    breakatwhitespace=false,
    breaklines=true,
    captionpos=b,
    keepspaces=true,
    % numbers=left,
    numbersep=5pt,
    showspaces=false,
    showstringspaces=false,
    showtabs=false,
    tabsize=2
}
\lstset{style=mystyle}





\begin{document}
\rmfamily


\begin{minipage}{\linewidth}
    \maketitle

    \sffamily
    \tableofcontents
\end{minipage}
\vspace{30pt}
\hrule
\vspace{1pt}





\section{Preface}

在这个时代,科学上网处于某种暧昧的地位。政治气候上,这属于舆论管理;司法实践上,这属于国际互联网信道管理。
本文探讨一种新型思路,对科学上网问题提供一种补充,规避科学上网实践牵扯到国际互联网信道问题。



\section{法理辨析}

关于国际互联网信道的定义,目前尚无正式的立法解释、司法解释。
这种模糊为公安机关、检察机关、基层法院提供了过大的自由裁量空间。
社会上一般认为,光缆、卫星信号才能构成国际互联网信道。
但是,在目前的司法实践中,使用 PPTP、L2TP、IPSec、Shadowsocks 等技术实现的 TCP/UDP 网络流量代理,可以被认定为国际互联网信道。

所以,某些信息的传达需要以另类的手段完成,境内设备和用户不能主动发起「虚拟信道」的创建。



\section{参与方面}

Wishnet 的工作方式需要多方协作。

\begin{compactitem}
	\item 请求者 Wisher
	\item 服务者 Dawei
\end{compactitem}

\subsection{请求者 Wisher}

请求者(Wisher)类似于传统 web 意义上的客户端(client)。但不同之处在于,请求者不会主动发起网络连接,只会公示自己的信息需求,期待服务者上传。

\subsection{服务者 Dawei}

服务者(Dawei)类似于传统 web 意义上的服务端(server)。但不同之处在于,服务者并非被动接受连入的网络连接,而是会主动查询请求者的信息需求,并尝试满足。




\section{信息需求的发布}

本章节介绍一种在 HTTP/HTTPS 协议栈上发布信息需求的方案。在此之外,也可以设计在其他协议栈(例如 rsyncd)上发布信息需求的方案。

请求者需要开设一个 HTTP 服务器。

\subsection{基本信息 Meta}

请求路径:\inlinecode{GET /meta.json}。

本文件包含请求者的一些基本信息。

内容样例:

\begin{lstlisting}
{
  // 一些基本信息
  "meta": {
    "hostname": "wishnet.example.com"
  },
  // 一些策略要求
  "policy": {
    "submitsPerMinute": 30
  }
}
\end{lstlisting}


\subsection{需求目录 Catalog}

请求路径:\inlinecode{GET /catalog.json}。

本文件为所有需求的目录。

内容样例:

\begin{lstlisting}
[
  // 数组中每个条目就是一个信息需求描述
  {
    // 唯一识别符
    "uuid": "c7c363d7-6e42-499e-9bbb-054ecceaf14c",
    // 信息被服务者获取的时间不早于某个阈值
    "after": "2021-01-01T12:00:00.000Z",
    // 规则描述
    "match": [
      // 一个规则
      {
        // 匹配模式:URL 字符串匹配
        "mode": "url-substring",
        // 以特定字符串开头和结尾,null 表示无约束
        "begin": "https://twitter.com/neruthes/status/",
        "end": null
      },
      // 另一个规则
      {
        // 匹配模式:URL 正则表达式
        "mode": "url-regexp",
        "regexp": "/^https\\:\\/\\/twitter.com\\/neruthes\\/status/\w+"
      }
    ]
  }
]
\end{lstlisting}












\section{服务者的上传服务}

本章节介绍一种在 HTTP/HTTPS 协议栈上发布信息需求的方案。在此之外,也可以设计在其他协议栈(例如 sftp、ssh、smtp)上发布信息需求的方案。

\subsection{提交数据}

请求路径:\inlinecode{POST /api/submit/(uuid)}。

通过 HTTP/HTTPS 提交数据,需要携带这些 header:

\begin{lstlisting}
// 原始的 URL
Wishnet-Raw-URI: https://twitter.com/neruthes/status/1489603911795482630
// 服务者缓存到这份资源的时间
Wishnet-Fetch-Date: Sat, 05 Feb 2022 13:56:43 GMT
// 原始的 MIME type
Wishnet-Content-Type: text/html
// 使用服务者的公钥,对 request body 做一份数字签名
Wishnet-Signature: iHUEARMIAB0WIQT6W9hcOY4NkrKGQ7U27uwNvkiaTQUCYf6EIQAKCRA27uwNvkia
    TVwUAP9OpFSWphREgoTNsW2t0ot1ga97jBDDxwM4PYia4KR3BAEAuFtVL7rd0nG6
    8ymIfzwXtpxfurq7VeM8NMNK1PM7aPI=
    =zAry
\end{lstlisting}

HTTP request body 即为服务者当时获取的 HTTP response body。




\section{请求者的信息管理策略}
// TODO

\subsection{多用户问题}
\subsection{面向用户的信息呈现}
\subsubsection{作为一个 Web Archive 服务}
\subsubsection{作为一个 HTTP Proxy 服务}
\subsection{阅读状态管理}
\subsection{老旧数据的备份与清理}
\subsection{服务者可靠性问题}
\subsubsection{公钥白名单模式}
\subsubsection{反向 DNS 域名白名单模式}

\section{信息需求目录的自动化生成}
// TODO

\section{服务者的工作策略}
// TODO


\end{document}