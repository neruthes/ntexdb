\documentclass[a4paper,11pt]{article}
\documentclass[a4paper,11pt]{article}
\usepackage{titling}
\usepackage{enumerate,enumitem,paralist}

% Use letterhead1 library
\input{/home/neruthes/DEV/ntexlibs/lib/letterhead1.tex}








% Fonts
\setmainfont{NewComputerModern}
\setromanfont{NewComputerModern}
\setsansfont{Inter}
\setmonofont{JetBrains Mono NL}
\setCJKmainfont{FandolSong}
\setCJKromanfont{FandolSong}
\setCJKsansfont{Noto Sans CJK SC}
\setCJKmonofont{Noto Sans CJK SC}

% Use alternative geometry
\newgeometry{textwidth=42em,tmargin=45mm,bmargin=33mm}

% % Paragraph style
% \setlength{\parindent}{2em}
% \setlength{\parskip}{4pt}

\newcommand{\Nmaketitle}[1]{%
    \begin{minipage}{\linewidth}%
        \parskip=6pt%
        \center%
        \rmfamily%
        \fontspec[Ligatures=TeX]{Latin Modern Roman}%
        {\LARGE#1}\par%
        {\sffamily\normalsize\theauthor}\par%
        {\sffamily\normalsize\thedate}%
    \end{minipage}\par
    \vspace{15pt}\hrule%
    \vspace{15pt}%
    \renewcommand{\ifootcornerL}[0]{#1}%
}
\newcommand{\Nmaketoc}[0]{\vspace{20pt}\hrule\tableofcontents\vspace{20pt}\hrule}


\renewcommand{\letterfooterfont}[0]{\sffamily\footnotesize}
\renewcommand{\ifootcornerL}[0]{A Note by \theauthor}
\renewcommand{\ifootcornerR}[0]{Page \thepage}

\newcommand{\icode}[1]{\texttt{\footnotesize#1}}


\fontsizepteleven
% \fontsizeptwelve
\docsetup{Secure Internet-Relayed Chat Protocol Set Architecture}{Neruthes}{\today, Early Draft, Version 0.1.0}
\linespread{1.05}
\frenchspacing






\begin{document}

\fulldoctitle


\section*{Abstract}

This document describes the design of the Secure Internet-Relayed Chat (SIRC) protocol set, on a manifesto level.
The SIRC protocol set is supposed to solve security problems in IRC (and similar IM solutions)
and usability problems in OpenPGP (and similar cryptography solutions)
with end-to-end encryption over public-key cryptography.

The SIRC protocol set includes several components (known as `layers'), each being independent from others. Namely:

\begin{compactitem}
	\item Identification Layer
	\item Discovery Layer
	\item Messaging Layer
	\item Carrier Layer
\end{compactitem}

This document may contain code snippets for software algorithm or data payload.
These snippets are provided for demonstration purposes only and should not be considered requirements on the specification level unless otherwise noted.
Protocol specifications shall be discussed in separate documents, or in a later version of this document.

\clearpage
\tableofcontents


\section{Identification Layer}

Alice, a user of the SIRC protocol set, has 2 keypairs: the Signature keypair (Alice.sig) and the Encryption keypair (Alice.enc).

\subsection{Signature Keypair}
The Signature keypair is comparable to the concept `master key' in OpenPGP.
In the eyes of other people, Alice is primarily identified by its Signature public key (Alice.sigPub).

In human-friendly environments, Alice.sigPub is usually addressed by its hash
(SHA-256 from OpenSSH private key format, hexadecimal representation, initial 40 digits, preferably uppercase),
which is known as the Fingerprint of Alice (Alice.fp).
The full hash of it is known as the Pawprint of Alice (Alice.pp).

Acceptable algorithms include: Ed25519.

\subsection{Encryption Keypair}
Unlike OpenPGP, Alice only has one Encryption keypair.

In the event of loss or compromise of the Encryption key,
Alice is encouraged to generate a new identity (Alice.sig and Alice.enc), instead of modifying the existing relation
between the Signature key and the Encryption key.

Acceptable algorithms include: X25519.






\section{Discovery Layer}

\subsection{SelfCert}
Alice may generate a SelfCert (Alice.cert) which may be used by Bob to add certain information of Alice
into its database for communicating with Alice in future.

A SelfCert shall be a UTF-8 plaintext file.

The SelfCert of Alice includes the following information:

\begin{compactitem}
	\item Timestamp
	\item Name of Alice
	\item Alice.fp
	\item Alice.sigPub
	\item Alice.encPub
	\item Supplementary information as specified in sections below
	\item Digital signature by Alice.sigPriv
\end{compactitem}

\subsection{SelfCert Relay Service}
A third party may operate a SelfCert Relay Service (SCRS).
A SCRS is an open database which returns a SelfCert when accepting a valid query word.
A query word may be a Fingerprint, a substring of a Fingerprint, or an allocated-through-registration human-friendly string (`alice').
A SCRS should serve over HTTP/HTTPS.
The support for querying by Fingerprint is mandatory, while the support for other categories of query words is optional.

For a query like `GET /u/alice',
it either returns the plaintext SelfCert, or returns a human-friendly webpage
which contains a <link> tag (rel=``selfcertplaintext'') to the plaintext SelfCert in its <head>.
In this case, Alice may simply write `https://example.com/u/alice' somewhere and
Bob (or the user agent of Bob) will be able to obtain Alice.cert from the SCRS.

A SCRS may broadcast SelfCert update events; it may also subscribe to such broadcasts.
If another SCRS has a newer version of the SelfCert of any known user (usually identified by its Fingerprint),
the listening SCRS may replace the old one with the new one.
Therefore the user itself may be relieved from making uploads in the event of updates.






\section{Messaging Layer}

% The Messaging layer consists of 4 sublayers.

% To send a piece of text from the Sender (Alice) to the Recipient (Bob).

\subsection{Main Workflow}
\newcounter{istepnum}
\setcounter{istepnum}{1}
\newcommand{\istep}[0]{Step \theistepnum: \stepcounter{istepnum}}
\newcommand{\istepend}[0]{\setcounter{istepnum}{1}}

\subsubsection{Steps}

\istep ContentBlob = TypeInfo + RawData

\istep Msg = ContentBlob + Alice.fp + ChatID + LastMsgPtr + Timestamp

\istep MicroSig = Sign( EphemKey.priv, Msg )

\istep CipherBlob = SymmetricEncrypt( MsgIV, MsgKey, Msg + MicroSig )

\istep RsKeys[ ] = Recipients[ ] map X $\rightarrow$ ECDH( EphemKey.priv, X.pub )

\istep RKCiphers[ ] = RsKeys[ ] map X $\rightarrow$ SymmetricEncrypt( MsgIV, X, MsgKey )

\istep SlotID = Hash( LastMsgPtr + ChatID )

\istep Header = RKCiphers[ ] + MsgIV + EphemKey + SlotID

\istep Envelope = Header + CipherBlob

\istepend

\subsubsection{Explanations}

\begin{description}
    \item[Data Message] What users want to send to each other.
    \item[Control Message] What the softwares need to send and receive in order to maintain good communication experiences.
    \item[TypeInfo] Metadata for the data being transported. Control messages are identified with respective types.
    \item[RawData] The actual data for transportation, usually text or image.
    \item[ChatID] The identifier of the current Conversation. Defined later.
    \item[SlotID] If Alice knows the ChatID of a message, it can try fetching the `next' messages after it, by calculating the SlotID of the next messages. A Message Relay Server (MRS) may maintain a key-value storage which maps SlotIDs to Envelopes. In other words, the participants of a Conversation keep opportunistically `jumping through various channels' (in Wi-Fi terms). See notes below for exceptions.
    \item[LastMsgPtr] The SHA-256 hash of the Msg payload of the last data message in the conversation available to the user agent of the sender of the current message. This is helpful for ordering messages chronically.
    \item[MicroSig] A deniable authentication mark. To Bob, Alice proves that the message is created by either Alice or Bob; to other people, Bob cannot prove that it is created by Alice. (A better technique for deniable authentication may replace this dumb one.)
    \item[SymmetricEncrypt] Encryption algorithm. Arguments: IV, key, data. Acceptable algorithms: AES-256, AES-512, ChaCha20-Poly1305, SM4.
    \item[RsKeys] Array of recipient-specific ciphers which the MsgKey can be decrypted from.
    \item[ECDH] The Elliptic-curve Diffie-Hellman algorithm.
    \item[EphemKey] An ephemeral key. See `ECDHE' for more information.
\end{description}

\subsubsection{Notes}

\begin{compactitem}
    \item The Envelope shall have a plaintext encoding format in addition to a binary format. So that it may be exchanged over text transportation solutions such as email and IRC.
    \item In order to avoid unnecessary jumping: if the new SlotID does not end with 0, the existing SlotID should be reused; if it does, we call this event a slotshift.
    \item If unlucky, Bob may miss a message and fail to catch up the chain of messages (Envelopes getting removed and inaccessible by their SlotID). In such occasion, Bob's next message to Alice would include a rather old LastMsgPtr. The user agent of Alice will be smart enough to notice this anomaly (e.g. the Conversation has moved 2 slotshifts away from that pointer), and will hence give Bob a snapshot of the messages (array of Envelopes) that he missed (with a large control message).
    \item Bob.pp is also a valid SlotID. The user agent of Bob shall always subscribe to this slot in order to know that someone may be sending a `cold' message (initiating a Conversation). When Alice sends a cold message, the user agent of Alice shall send a `malformed' Envelope (in addition to the good one) whose SlotID is `wrongly' set to Bob.pp. It is left for the user, not the user agent, to decide how to treat incoming cold messages.
\end{compactitem}


\subsection{Conversation Management}
A Conversation may be an Unnamed Conversation or a Named Conversation.

\subsubsection{Unnamed Conversation}
An Unnamed Conversation is identified by its ChatID, the SHA-256 hash of the sorted list (smaller on top, delimited by colon) of the Fingerprints of its members.

$$
	ChatID = hash( Fp_{1} , Fp_{2} , Fp_{3} , \ldots )
$$

\subsubsection{Named Conversation}
A Named Conversation is similar to `chat group' (Telegram) or `channel' (IRC).

// TODO












\section{Carrier Layer}

\subsection{Message Relay Server}

A Message Relay Server (MRS) may facilitate the transportation and relaying for the encrypted message payload (`Envelope').
The user agent of Alice maintains a list of MRSs (built in from source code, or specified by distro maintainers / system admin / end user).

When the aforementioned Envelope is ready for sending, Alice (its user agent) submits the Envelope to some (or all) trusted MRSs.
If Alice and Bob are lucky, i.e. Bob happens to subscribe to incoming message notifications (IMNs) from any of the trusted MRSs,
Bob actually subscribes to increasingly more SlotIDs on the MRSs.

MRSs may voluntarily exchange Envelopes with each other, since Envelopes are already highly anonymized.

\subsection{SelfCert Extension: MRSList}

In the SelfCert, Bob may add an optional data entry: MRSList.
If this list is present and known to Alice, Alice shall also upload the Envelope to all the MRSs listed here besides uploading to its own trusted MRSs.





\section{Afterwords}

This document is a rather early draft of a rather high-level manifesto.
The author looks forward to comments on the macro architecture and the major concepts.
If a reader would like to offer feedbacks, they may be directed to the author. Discussions are welcomed.

\end{document}
