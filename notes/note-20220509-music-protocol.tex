\documentclass[a4paper,11pt]{article}
\usepackage{tocloft,tabu}

% Use letterhead1 library
\input{/home/neruthes/DEV/ntexlibs/lib/letterhead1.tex}

\usepackage{amssymb}

\usepackage{paralist,enumerate,enumitem}





% Fonts
\setmainfont{NewComputerModern}
\setromanfont{NewComputerModern}
\setsansfont{Inter}
\setmonofont{JetBrains Mono NL}
\setCJKmainfont{FandolSong}
\setCJKromanfont{FandolSong}
\setCJKsansfont{Noto Sans CJK SC}
\setCJKmonofont{Noto Sans CJK SC}

% Use alternative geometry
\renewcommand{\letterfooterwidth}[0]{44em}

\newcommand{\refreshgeometry}[0]{
    \newgeometry{a4paper,textwidth=\letterfooterwidth,tmargin=45mm,bmargin=29mm}
}
\newcommand{\fontsizepteleven}[0]{
    \renewcommand{\letterfooterwidth}[0]{40em}
    \refreshgeometry
}
\newcommand{\fontsizeptwelve}[0]{
    \renewcommand{\letterfooterwidth}[0]{37em}
    \refreshgeometry
}

% % Some dimensions
\setlength{\tabulinesep}{3pt}
% \setlength{\parindent}{2em}
% \setlength{\parskip}{4pt}

\newcommand{\simpledoctitle}[0]{%
    \begin{minipage}{\linewidth}%
        \parskip=6pt%
        \center%
        \rmfamily%
        \fontspec[Ligatures=TeX]{Latin Modern Roman}%
        {\LARGE\cacheddoctitlebig}\par\vspace{4pt}%
    \end{minipage}\par
    \vspace{15pt}\hrule%
    \vspace{15pt}%
    \renewcommand{\ifootcornerL}[0]{\cacheddoctitlefooter}%
}
\newcommand{\fulldoctitle}[0]{%
    \begin{minipage}{\linewidth}%
        \parskip=6pt%
        \center%
        \rmfamily%
        \fontspec[Ligatures=TeX]{Latin Modern Roman}%
        {\LARGE\cacheddoctitlebig}\par\vspace{4pt}%
        {\sffamily\normalsize\cacheddocauthor}\par%
        {\sffamily\normalsize\cacheddocdate}%
    \end{minipage}\par
    \vspace{15pt}\hrule%
    \vspace{15pt}%
    \renewcommand{\ifootcornerL}[0]{\cacheddoctitlefooter}%
}
\newcommand{\setrealdisplaytitle}[1]{
    \renewcommand{\cacheddoctitlebig}[0]{#1}
    \renewcommand{\cacheddoctitlefooter}[0]{#1}
}
\newcommand{\Nmaketoc}[0]{\vspace{20pt}\hrule\tableofcontents\vspace{20pt}\hrule}


\renewcommand{\letterfooterfont}[0]{\sffamily\footnotesize}
\renewcommand{\ifootcornerL}[0]{A Note by \theauthor}
\renewcommand{\ifootcornerR}[0]{Page \thepage}

\newcommand{\icode}[1]{\texttt{\footnotesize#1}}



\newcommand{\Asection}[1]{
    \section*{#1}
    \addcontentsline{toc}{section}{#1}
}
\newcommand{\Asubsection}[1]{
    \subsection*{#1}
    \addcontentsline{toc}{subsection}{#1}
}



\geometry{textwidth=40em}
\docsetup{Music Streaming Service Protocol Architecture Design}{Neruthes \& Catten Linger}{2022-05-09}
\linespread{1}







\begin{document}

\fulldoctitle






\section*{Abstract}

This protocol is designed to enable a distributed network for music stream service,
where everyone may host their own music inventories and exchange access with each other.










\Nmaketoc\clearpage












\section{Roles}

% \subsection{Roles Overview}

% Our architecture consist of the following roles:

% \begin{tabu}{|l|X|}
% 	\hline
% 	{Role}                           & {Description}                                                                   \\
% 	\hline
% 	{Inventory Server}               & {Hosts music files through HTTP/HTTPS.}                                         \\
% 	{Index Server}                 & {Maintains an index of music tracks from various inventory servers.}           \\
% 	{Player Server: Web Player}      & {Follows index servers and provide in-browser playing service for end users.} \\
% 	{Player Server: Independent App} & {Follows index servers and plays tracks.}                                     \\
% 	{Subscriber Client}              & {Subscribes to a player server and plays tracks.}                               \\
% 	\hline
% \end{tabu}


\subsection{Inventory Server}

Responsibilities:

\begin{compactitem}
	\item Serves music files through HTTP/HTTPS.
	\item Maintains a self-declaration file for information of this inventory.
	\item Maintains an index for available music files with metadata (ID3 tags).
\end{compactitem}





\subsection{Index Server}

% A index server shall maintain a list of upstream inventory servers (InvSerList).
% By collecting their respective local catalogs, the index server shall maintain individual counters.

Responsibilities:

\begin{compactitem}
	\item Maintains a list of upstream inventory servers.
	\item Periodically collects upstream catalogs to build an aggregated catalog.
	\item Disambiguation for name collisions among artists.
	\item Decides whether the tracks (in different albums) sharing the same artist name are really performed by the same artist.
	\item Uniquely identifies each album.
	\item (Optional) Selectively applies patches to track metadata, e.g. consolidate artist names.
	\item (Optional) Uses rules to mask certain tracks from an inventory server from being indexed.
\end{compactitem}

Notes:

\begin{compactitem}
	\item Each album should be identified by the Cartesian product of album artist and album title.
\end{compactitem}














\subsection{Player Server: Web Player}

Responsibilities:

\begin{compactitem}
	\item Follows index servers and provide in-browser playing service for end users
\end{compactitem}

\subsection{Player Server: Independent App}

Responsibilities:

\begin{compactitem}
	\item Follows index servers and plays tracks.
\end{compactitem}

\subsection{Subscriber Client}

Responsibilities:

\begin{compactitem}
	\item Subscribes to a player server and plays tracks.
\end{compactitem}






















\section{API}


\subsection{Inventory Server APIs}

\subsubsection{The MusicSite File}

This is a JSON file at \icode{\$siteurl\_main/.MusicSite.json}.

The file shall include the following properties:

\begin{tabu}{|l|X|}
	\hline
	{Property}      & {Description}                                                       \\
	\hline
	{title}         & {Human-readable title of this inventory.}                           \\
	{siteurl\_main} & {Canonical URL of this inventory.}                                  \\
	{siteurl}       & {Array of alternative URLs.}                                        \\
	{peers}         & {(Optional) Array of other inventories which the owner recommends.} \\
	\hline
\end{tabu}

Example:

\begin{lstlisting}
{
    "title": "Example Music Site",
    "siteurl_main": "https://music.example.com/music/",
    "siteurl": [
        "https://music.example.com/music/",
        "http://10.0.233.10:12345/disk2/audio/music/",
        "ftp://10.0.233.10/music/"
    ],
    "peers": [
        "https://anothermusicservice.com/"
    ]
}
\end{lstlisting}

\subsubsection{The MusicCatalog File}

This is a JSON file at \icode{\$siteurl\_main/.MusicCatalog.json}. An array of tracks.

Each track shall include the following properties:

\begin{tabu}{|l|X|}
	\hline
	{Property} & {Description}                                                                  \\
	\hline
	{path}     & {Relative path from this catalog file.}                                        \\
	{size\_KB} & {Size of this music file in KB (1 KB = 1024 bytes).}                           \\
	{album}    & {Album name.}                                                                  \\
	{title}    & {Track title.}                                                                 \\
	{artist}   & {Track artist.}                                                                \\
	{disc}     & {(Nullable) Disc number if the album has multiple discs, starting from ``1''.} \\
	{track}    & {Sequential position of the track in the disc, starting from ``01''.}          \\
	\hline
\end{tabu}

Example:

\begin{lstlisting}
[
    {
        "path": "Genshin/01_风与异乡人/01 原神.flac",
        "size_KB": "19704",
        "album": "风与异乡人",
        "title": "原神",
        "artist": "陈致逸/HOYO-MiX",
        "disc": "",
        "track": "01"
    }
]
\end{lstlisting}






\subsection{Index Server APIs}


\newcommand{\restfulapi}[3]{
	\fbox{
        \begin{minipage}{0.9\linewidth}
            \parskip=7pt
            \vspace{5pt}
            % $\blacksquare$ \textsf{#1}
            $\blacksquare$ {#1}
            
            Endpoint: \icode{#2}
            
            \textbf{Note}:\hspace{1em}#3
            \vspace{5pt}
        \end{minipage}
    }
}


\subsubsection{Entity APIs}

% \subsubsection{Track}

\restfulapi{Get Track Info}{GET /api/entity/track/[TrackID]}{None.}

% \subsubsection{Album}

\restfulapi{Get Album Info}{GET /api/entity/album/[AlbumID]}{None.}

% \subsubsection{Artist}

\restfulapi{Get Artist Info}{GET /api/entity/artist/[ArtistID]}{None.}

\subsubsection{Search APIs}

\restfulapi{Wild Search}{GET /api/search?query=[Criteria]}{Use this API to search in all fields}

\restfulapi{Scoped Search}{GET /api/search?query=[Criteria]\&scopes=[Scope1,Scope2]}{Use this API to search in specified ranges, where valid scopes include `title', `album', and `artist'.}







% \section{Architecture}









\end{document}
