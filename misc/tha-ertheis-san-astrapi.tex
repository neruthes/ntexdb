%!TEX TS-program = xelatex
%!TEX encoding = UTF-8 Unicode

\documentclass[a4paper,11pt]{article}


\newcommand{\recipientaddress}{
    \begin{minipage}{37.99mm}
        \begin{FlushRight}
            % \raggedright
            \sffamily
            \mdseries
            \footnotesize
            \setlength{\baselineskip}{14pt}
            \setlength{\parskip}{0pt}

            % 总务司\par
            % 璃月港

        \end{FlushRight}
    \end{minipage}
}

\input{/home/neruthes/DEV/ntexlibs/lib/letterhead1.tex}
% \input{/home/neruthes/DEV/ntexlibs/lib/letterhead2.tex}

% \usepackage{showframe}
\usepackage{lipsum}

\begin{document}
\pagestyle{plain}
\rmfamily\normalsize
\setlength{\parskip}{7pt}
\setlength{\baselineskip}{15pt}
\raggedright
\raggedbottom

\normalsize

Θά 'ρθεις σαν αστραπή\\
Θά' χει η χώρα γιορτή\\
Θάλασσα γη και ουρανός\\
Στο δικό σου φως

Θα ντυθώ στα λευκά\\
Να σ' αγγίξω ξανά\\
Φως εσύ και καρδιά μου εγώ\\
Πόσο σ' αγαπώ

Βασιλεύς Βασιλέων Βασιλεί Βοήθει\\
Έλεος, έλεος Επουράνιε Θεέ\\
Κωνσταντίνος Δραγάτσης Παλαιολόγος\\
Έλεω Θεού Αυτοκράτωρ των Ρωμαίων

Στην πύλη του αγίου Ρωμανού\\
Καβαλικά την φάρα του την ασπροποδαράτην\\
Τέσσερα Βήτα, έλεος, έλεος\\
Μαρμαράς Βόσπορος και Μαύρη Τρίτη

Φρίξον ήλιε, στέναξον γη\\
Εάλω ή πόλη, Εάλω η πόλη\\
Βασιλεύουσα, πύλη χρυσή\\
Κι ο πορφυρογέννητος στην κόκκινη μηλιά

Η πόλη ήταν το σπαθί, η πόλη το κοντάρι\\
Η πόλη ήταν το κλειδί της Ρωμανίας όλης\\
Σώπασε Κυρά Δέσποινα και μην πολυδακρύζεις\\
Πάλι με χρόνια με καιρούς, πάλι δικά Σου θά ναι

Στην πύλη του αγίου Ρωμανού\\
Έφυγες για αλλού\\
Κι άγγελος θα σε φέρει εδώ\\
Στον σωστό καιρό

Μες την Άγια Σοφιά\\
Θα βρεθούμε ξανά\\
Λειτουργία μελλοντική\\
Οι Έλληνες μαζί

\end{document}
