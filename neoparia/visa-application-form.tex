%!TEX TS-program = xelatex
%!TEX encoding = UTF-8 Unicode

\documentclass[a4paper,10pt]{article}

\usepackage[a4paper,textwidth=48em,vmargin=18mm]{geometry}
\usepackage{calc}
\usepackage{amsmath,xltxtra,fontspec,xunicode}
\usepackage{titlesec,datetime2}
\usepackage{enumitem,paralist,tabu}
\usepackage[OT1]{fontenc}
\usepackage{wasysym,oz,pxfonts,txfonts}
\usepackage{ragged2e}

\usepackage[PunctStyle=plain,RubberPunctSkip=false]{xeCJK}
\defaultCJKfontfeatures{Script=CJK}
\XeTeXlinebreaklocale "zh"
\XeTeXlinebreakskip = 0pt plus 1pt
\setmainfont{Noto Serif CJK SC}
\setromanfont{Noto Serif CJK SC}
% \setsansfont{Inter}
\setsansfont{Noto Sans CJK SC}
\setmonofont{JetBrains Mono NL}
\setCJKmainfont{Noto Serif CJK SC}
\setCJKromanfont{Noto Serif CJK SC}
\setCJKsansfont{Noto Sans CJK SC}
\setCJKmonofont{Noto Sans CJK SC}

\usepackage[dvipsnames]{xcolor}
\usepackage{graphicx}
\graphicspath{ {/home/neruthes/DEV/ntexlibs/pic} }

\usepackage{tasks}


\setlength{\parindent}{0pt}
\setlength{\parskip}{5pt}
\setlength{\tabulinesep}{11pt}

\linespread{1.1}




\newcommand{\inforow}[2]{
    \begin{minipage}{\linewidth}
        \vspace{3pt}
        \infocell{#1}{#2}\\
        \vspace{3pt}
        \hrule
        \vspace{5pt}
    \end{minipage}\\
}
\newcommand{\inforowDuo}[4]{
    \vspace{3pt}
    \begin{minipage}{0.5\linewidth-4pt}
        \infocell{#1}{#2}
    \end{minipage}
    \hfill
    \begin{minipage}{0.5\linewidth-4pt}
        \infocell{#3}{#4}
    \end{minipage}\\
    \vspace{3pt}
    \hrule
    \vspace{5pt}
}
\newcommand{\infocell}[2]{
    % $1=zh  $2=en
    \begin{minipage}{\linewidth}
        #1\\
        {\fontsize{7.5pt}{8pt}\selectfont#2}
    \end{minipage}
}
\newcommand{\inforowbig}[4]{
    % $1=zh  $2=en  $3=zhnote  $4=ennote
    \vspace{3pt}
    \infocell{#1}{#2}\\
    \vspace{3pt}
    \hrule
    \vspace{2pt}
    {\fontsize{8pt}{10pt}\selectfont#3 {\fontsize{7.5pt}{10pt}\selectfont#4}}
    \vspace{7pt}\par
}







\begin{document}
\sffamily
\pagestyle{empty}

\begin{minipage}{\linewidth}
	\center
    \rmfamily
    \fontspec{Playfair Display}
    \scshape
	\Large
	% \bfseries
	纽巴尼亚共和国签证申请表\\
    Neoparia Demokratia Visa Application Form
	% NEOPARIA DEMOKRATIA VISA APPLICATION FORM

	\vspace{4pt}
    \ttfamily
	\normalsize
	Latest Revision \today
\end{minipage}

\vspace{10pt}
\hrule
\vspace{5pt}

\begin{minipage}{\linewidth}
	请使用简体中文或美式英文填写申请表。\\
	Please fill in English (United States) or Chinese (Simplified).
\end{minipage}


\fontsize{9pt}{10pt}\selectfont

\section{基本信息 / Basic Information}

\inforowbig{标准名称}{Canonical Name}{请使用贵国官方语言的正字法。}{Please follow the orthography of the official language of your country.}
\inforowbig{ASCII 拉丁字母名称}{ASCII-Compatible Name}{请使用 ASCII 标准中的 26 个拉丁字母。}{Please use the 26 letters in the ASCII alphabet.}
\inforow{护照姓}{Surname on Passport}
\inforow{护照名}{Given Name on Passport}
\inforowDuo{护照号码}{Passport No}{出生年月日 (YYYY-MM-DD)}{Date of Birth (YYYY-MM-DD)}
\inforowDuo{国籍}{Nationality}{出生国籍}{Nationality at Birth}

\section{旅行信息 / Travel Information}

\inforowbig{签证类型}{Visa Class}{类型细节请参阅帮助文档。}{Please consult the manual for detailed class definitions.}

\begin{tasks}[label=,after-item-skip=0pt,item-indent=0pt](2)
    \task A1: 外交常驻 Persistent Diplomat
    \task A2: 外交临时 Temporary Diplomat
    \task B1: 商务类访问 Business Visit
    \task B2: 个人类访问 Personal Visit
\end{tasks}


\end{document}
