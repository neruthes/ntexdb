\input{/home/neruthes/DEV/ntexlibs/lib/paper-academic.tex}

\usepackage{multicol}


\begin{document}
\imaketitle{Multilateralist Country Status Recognition Scheme \linebreak as a Diplomacy Card for China Government to Play}{Neruthes}{[Early Preview Draft]}

\section*{Abstract}

At least since 2018, multilateralism has become an important diplomatic stance for China Government to perform. While the People's Republic of China has a rather longer history of favoring multilateralism from government to civilians, not before the Trump Trade War did China Government become a vocal advocate for the ideology on the international stage.

This article will explore how China Government may substantiate the ideology and put it into the policy toolbox by introducing a Multilateralist Country Status Recognition Scheme.

\maketoc{\small}{\clearpage}

\section{A Brief History of Market Economy Status Recognition}
% \begin{multicols}{2}
Historically, White House had been playing the Market Economy card with China, using the status recognition issue to blockade certain market areas in the United States from Chinese companies, especially the State-Owned Enterprises (SOEs). This card had been played well even after China joined the WTO, as a preplanned countermeasure for any potential market access rejection on the Chinese side, or simply as a mean of blackmailing.

Along with human rights dramas and other stuff, White House had made up a comprehensive trade policy toolbox in terms of confronting China.
% \end{multicols}

\section{Pushing the Downfall of Hegemony}
\section{Designing the Criteria}
\subsection{How to Exclude the United States}
\subsection{Rallying the Third World}
\subsection{A Possible List of Criteria}
\section{Differentiated Treatment for Multilateralist Country}
\subsection{Trade Policy}
\subsection{Monetary Policy}
\subsection{Fiscal Policy}
\subsection{Security Policy}
\section{Interactions with Other Diplomacy Cards}
\section{Interactions with Domestic Political Agenda}

\end{document}
