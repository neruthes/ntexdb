%!TEX TS-program = xelatex
%!TEX encoding = UTF-8 Unicode

\documentclass[a4paper,10pt]{article}

\usepackage[a4paper,textwidth=48em,vmargin=18mm]{geometry}
\usepackage{calc}
\usepackage{amsmath,xltxtra,fontspec,xunicode}
\usepackage{titlesec,datetime2}
\usepackage{enumitem,paralist,tabu}
\usepackage[OT1]{fontenc}
\usepackage{wasysym,oz,pxfonts,txfonts}
\usepackage{ragged2e}

\usepackage[PunctStyle=plain,RubberPunctSkip=false]{xeCJK}
\defaultCJKfontfeatures{Script=CJK}
\XeTeXlinebreaklocale "zh"
\XeTeXlinebreakskip = 0pt plus 1pt
\setmainfont{Noto Serif CJK SC}
\setromanfont{Noto Serif CJK SC}
% \setsansfont{Inter}
\setsansfont{Noto Sans CJK SC}
\setmonofont{JetBrains Mono NL}
\setCJKmainfont{Noto Serif CJK SC}
\setCJKromanfont{Noto Serif CJK SC}
\setCJKsansfont{Noto Sans CJK SC}
\setCJKmonofont{Noto Sans CJK SC}

\usepackage[dvipsnames]{xcolor}
\usepackage{graphicx}
\graphicspath{ {/home/neruthes/DEV/ntexlibs/pic} }

\setlength{\parindent}{0pt}
\setlength{\parskip}{5pt}
\setlength{\tabulinesep}{11pt}

\linespread{1.1}








\begin{document}
\sffamily
\pagestyle{empty}

\begin{minipage}{\linewidth}
	\center
	\Large
	% \bfseries
	纽巴尼亚共和国签证申请表\\
	NEOPARIA DEMOKRATIA VISA APPLICATION FORM

	\vspace{4pt}
	\normalsize
	Latest Revision \today
\end{minipage}

\vspace{10pt}
\hrule
\vspace{5pt}

\begin{minipage}{\linewidth}
	请使用简体中文或美式英文填写申请表。\\
	Please fill in English (United States) or Chinese (Simplified).
\end{minipage}



\section{基本信息 / BASIC INFO}


\fontsize{9pt}{9pt}\selectfont
\begin{minipage}{\linewidth}
	\begin{tabu} {|X|X[3]|}
		\hline
		{标准名称 Canonical Name}       & {}             \\
		\hline
		{护照姓 Passport Surname}       & {}             \\
		\hline
		{护照名 Passport Given Name}    & {}             \\
		\hline
		{护照号码 Passport No}          & {}             \\
		\hline
		% \end{tabu}
		% \begin{tabu} {|l|X|}
		% \hline
		{出生年月日 Date of Birth}      & {(YYYY-MM-DD)} \\
		\hline
		{国籍 Nationality}              & {}             \\
		\hline
		{出生国籍 Nationality at Birth \hfill Hello world} & {}             \\
		\hline
	\end{tabu}
\end{minipage}


\end{document}
