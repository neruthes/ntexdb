\documentclass[a4paper,11pt]{article}
\documentclass[a4paper,11pt]{article}
\usepackage{titling}
\usepackage{enumerate,enumitem,paralist}

% Use letterhead1 library
\input{/home/neruthes/DEV/ntexlibs/lib/letterhead1.tex}








% Fonts
\setmainfont{NewComputerModern}
\setromanfont{NewComputerModern}
\setsansfont{Inter}
\setmonofont{JetBrains Mono NL}
\setCJKmainfont{FandolSong}
\setCJKromanfont{FandolSong}
\setCJKsansfont{Noto Sans CJK SC}
\setCJKmonofont{Noto Sans CJK SC}

% Use alternative geometry
\newgeometry{textwidth=42em,tmargin=45mm,bmargin=33mm}

% % Paragraph style
% \setlength{\parindent}{2em}
% \setlength{\parskip}{4pt}

\newcommand{\Nmaketitle}[1]{%
    \begin{minipage}{\linewidth}%
        \parskip=6pt%
        \center%
        \rmfamily%
        \fontspec[Ligatures=TeX]{Latin Modern Roman}%
        {\LARGE#1}\par%
        {\sffamily\normalsize\theauthor}\par%
        {\sffamily\normalsize\thedate}%
    \end{minipage}\par
    \vspace{15pt}\hrule%
    \vspace{15pt}%
    \renewcommand{\ifootcornerL}[0]{#1}%
}
\newcommand{\Nmaketoc}[0]{\vspace{20pt}\hrule\tableofcontents\vspace{20pt}\hrule}


\renewcommand{\letterfooterfont}[0]{\sffamily\footnotesize}
\renewcommand{\ifootcornerL}[0]{A Note by \theauthor}
\renewcommand{\ifootcornerR}[0]{Page \thepage}

\newcommand{\icode}[1]{\texttt{\footnotesize#1}}


\fontsizepteleven
\docsetup{Model of Seven Layers for Cryptographic Communication Protocols}{Neruthes}{2023-02-12 (WIP)}







\begin{document}

\fulldoctitle


\section*{Abstract}

We aim to establish a loose suite of cryptographic communication protocols in a KISS fashion.
Every single protocol works on its own and connects to other protocols with clearly defined boundaries.

This document is a high-level overview and discusses the integration of protocols.
We may sometimes mention hypothetical protocols and describe their features;
such text pieces are not design specifications of the mentioned protocols,
but hints to offer rich understanding of how layers are divided and how
(hypothetical and existing) protocols may interoperate with each other.

% \clearpage
\tableofcontents\clearpage




\section{Layers}

Like the OSI model of computer networking, we must identify several layers for the cryptography ecosystem being designed.

Our Suite defines the following layers (from infrastructure to application):

\begin{description}
	\item[Data Delivery Layer (DDL)]
		Responsible for routing and delivery of arbitrary data cargos on top of it.
		May or may not break down cargos into packets, frames, etc.
	\item[Representation Formatting Layer (RFL)]
		Responsible for representing the data in a way (e.g. Base64) that the Data Delivery Layer can safely transport.
	\item[Room Management Layer (RML)]
		Client-server communication below E2EE.
		Rooms are similar to frequencies in terms of FM radio.
		Maintains distinguished rooms (e.g. IRC channels) in mansions (e.g. IRC servers).
		A room hosts a timeline of envelopes posted in the room, to be available for public retrieval;
		an expiry policy may optionally be applied to envelopes (by the owner of the mansion).
		Any posting in a room is public information.
		Hint: Protocols of higher levels can implement dynamic room selection algorithms.
	\item[Channel Management Layer (CML)]
		Similar to TLS.
		Maintains channels for sessions above it.
		A channel facilitates the communication for 2 parties (i.e. user devices).
		Provides E2EE security for layers above (SML to PIL).
		Requires an existing cipher key (i.e. created via a key negotiation procedure).
		Hint: Use 0-RTT key negotiation for establishing a default channel.
	\item[Session Management Layer (SML)]
		Maintains sessions for messages above it.
		A session may have a type and several participants (i.e. persons).
	\item[Message Packaging Layer (MPL)]
		Constructs and interprets self-declaring message payloads which indicate the content and the context.
	\item[Person Identification Layer (PIL)]
		May hardly be called a layer, but is important enough.
\end{description}





\section{Responsibilities of Layers}

\subsection{Data Delivery Layer (DDL)}

A DDL protocol provides cargo transportation capabilities over data railways.
A cargo is a binary data blob.

A data railway may be an IRC chat room, a Telegram chat, etc.
Also, a data railway may be a simple wrapper over stream protocols, e.g. TCP, WebSocket, WebRTC.

All layers above DDL are packet protocols and not stream protocols.
Realtime video/audio is tricky. We will discuss them later.

Cargo ordering is not a big concern for a DDL protocol, since SML protocols are expected to handle the ordering of messages.

\subsection{Representation Formatting Layer (RFL)}

A RFL protocol is responsible for representing the data in a way (e.g. Base64) that the DDL protocol can safely transport.

This layer is negligible when working with a DDL protocol which is capable of transporting verbatim binary data packets.
However, for DDL protocols which transport data over text-based applications (e.g. email, IRC),
it can be necessary to do some wrapping (e.g. Base64, Base32, Base4096Hangul) to ensure both human-friendliness and railway-compatibility.

Encoding is done at this layer so that DDL protocols can focus on their own job.

\subsection{Room Management Layer (RML)}

A RML protocol provides room selection algorithms so that the CML protocol running over it
may be able to connect to multiple rooms.

A room is an abstraction of the DDL protocol instance.
Rooms can be IRC chat rooms, Telegram chats, etc.

Rooms can be noisy so that unwanted data blobs may be observed.
A CML protocol shall do the filtering to identify the packets it wants when reading from rooms.

\subsection{Channel Management Layer (CML)}

A CML protocol provides a secure E2EE communication environment for SML protocol datagrams going through it.
It implements the security of communication between exactly 2 parties at the device level.

Each packet sent to the channel manager on the remote machine (RCM) should be an independent decryption unit ---
the RCM should be able to decrypt the packet without too much information beyond channel metadata.

The exchange of encrypted packets requires a room.
Packets may be sent into rooms and may be received from rooms.

While 0-RTT key negotiation SHOULD be preferable, round trips are acceptable.

Hint: A CML protocol can initialize the channel via a 0-RTT algorithm for the first outgoing SML datagram
with necessary metadata attached to negotiate another key.

A CML protocol has 2 responsibilities:

\begin{enumerate}
	\item Accepts SML datagrams from the session manager on the local machine
	      and encrypts them and sends encrypted secure boxes to the channel manager on the remote machine.
	\item Accepts encrypted secure boxes from the channel manager on the remote machine
	      and decrypts them and sends SML datagrams to the session manager on the local machine.
\end{enumerate}

A CML protocol MAY have the following features:

\begin{enumerate}
    \item Deniable authentication
\end{enumerate}

\subsection{Session Management Layer (SML)}

A SML protocol is responsible for determining the chronicle order of messages going through the session in either direction.

The quantity of participants of a session may range from 2 to some very large number (e.g. 1024).
Participants are persons (raw X.25519 keypairs, X.509 self-signed certificates, etc),
contrary to devices in CML.

Hint: When a session has too many participants, encryption is meaningless
(for mitigating a prepared spy, not an automated surveillance system) because everyone can leak.

A SML protocol consumes inputs and gives outputs,
and maintains internal states for any given session (e.g. identifier of the last incoming message).
The session instance on one machine interacts with the messaging manager on the local machine (LMM)
and the session manager on the remote machine (RSM).

Interactions with LMM:

\begin{itemize}
	\item Input:
	      \begin{itemize}
		      \item Real message package (outgoing)
	      \end{itemize}
	\item Output:
	      \begin{itemize}
		      \item Real message package (incoming)
	      \end{itemize}
\end{itemize}

Interactions with RSM:

\begin{itemize}
	\item Input:
	      \begin{itemize}
		      \item Message datagram
		      \item Control datagram
	      \end{itemize}
	\item Output:
	      \begin{itemize}
		      \item Message datagram
		      \item Control datagram
	      \end{itemize}
\end{itemize}

A datagram is a stand-alone self-declaring envelope of data which shall be transmitted securely through a CML protocol.
It is comparable to HTTP request.

A Message datagram is a wrapping over the message so that the chronicle position of the message may be determined by all participants.

A Control datagram is a hint for performance-improving and error-handling.

Possible use cases of Control datagrams include:

\begin{enumerate}
    \item Session initialization notification
    \item Hint to re-acquire my user metadata
    \item Change of online status (``Hey I am online!'')
    \item Message read receipt
    \item Request for retrieving historic messages
    \item Response for retrieving historic messages
    \item Broadcast for the identity of a newly invited member in a chat group
\end{enumerate}

\subsection{Message Packaging Layer (MPL)}

A MPL protocol constructs and interprets self-declaring message payloads which indicate the content and the context,
so that the messaging manager on the remote machine (RMM) can interpret messages and extract the contents for human interaction.
Messages may be structured and categorized.

Messages shall be piped into sessions and shall be received from sessions.
The exchange of messages between persons requires a session.

\subsection{Person Identification Layer (PIL)}

A PIL protocol defines how persons are identified/authenticated,
and provides cryptographic parameters to the CML protocol (for encryption and deniable authentication)
and to the MPL protocol (for digital signature).

Depending on the application scenario, a person may have an encryption keypair or a signature keypair, or both.

A PIL protocol SHALL define unique short identifiers for persons, e.g. fingerprint (hash of the public key).

A PIL protocol MAY define internal trust models. For example, OpenPGP defines 5 levels of trust and has a ``subkey'' architecture.




\section{Sample Application Scenarios}

\subsection{Cismail: Email-Like Messaging}

\href{https://pub-714f8d634e8f451d9f2fe91a4debfa23.r2.dev/keep/ntexdb/note-20230210-cismail.pdf--9f1bb76e5b1938be3c8907311238dbc0.pdf}{PDF Link}.

\subsection{IRC-Like Chatting}

\begin{enumerate}
	\item PIL: X.509 self-signed certificate with Ed25519/X.25519 keypairs
	\item MPL: Compact Multimedia Message Packaging Protocol (CMMPP)
	\item SML: Interactive Message Exchange Session Protocol (IMESP)
	\item CML: Secure E2EE Posting Channel Protocol (SEPCP)
	\item RML: Dynamic Chat Room Switching Protocol (DCRSP)
	\item RFL: Binary Modular Envelope Protocol (BMEP)
	\item DDL: IRC + Nostr
\end{enumerate}





\end{document}
