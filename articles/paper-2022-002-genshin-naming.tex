\input{/home/neruthes/DEV/ntexlibs/lib/paper-academic.tex}
\uselangzh

\usepackage{datetime2}
\usepackage{longtable}
\usepackage{tabu}
\usepackage{tasks}

\setlength{\tabulinesep}{5pt}


\begin{document}
\imaketitle{\huge 通用命名学}{Neruthes}{\today{} (ver 0.1.1)}

\section*{简介}

本文旨在介绍一种通用的遣词造句方案,方便玩家之间交流。本文所设计的方案,基于大量网友讨论总结而成,并非完全由作者原创。

\maketoc{}{}

\vspace{2em}
\hrule

\section{角色配装命名}

\subsection{示例}

\begin{itemize}
	\item 钟离 4 命,决斗精 4,绝缘 4,生岩暴伤,310。
	\item 90 枫原万叶 0 命,铁蜂刺精 1,风 4,精精精,999。
	\item 85 魈,和璞鸢精 1,风 2 散件,攻风暴伤,10210。
	\item 菲谢尔 1 命,85 阿莫斯精 1,苍白+骑士,攻物暴,10/11/1。
\end{itemize}

\subsection{组成结构}

\noindent\begin{tabu} {|l|X|}
	\hline
	{组件}         & {解释说明}                                                                                                                                                                                                        \\
	\hline
	{角色基本信息} & {写出名字、等级、命之座层数。可以省略等级,默认 90 级。可以省略命之座,默认 0 命。}                                                                                                                               \\
	{武器}         & {写出武器名字、精炼层数。可以省略等级,默认 90 级。可以省略精炼,五星武器默认精 1,四星武器默认精 5。}                                                                                                            \\
	{圣遗物套装}   & {四件套可以省略数字。两件套配散件可以省略散件。两个两件套需要用加号连接。五散件可以省略套装信息。}                                                                                                                \\
	{圣遗物主属性} & {写下后三件的主属性,均用主属性名字的第一个字。但有例外:元素充能效率简称「充」,元素精通简称「精」,暴击伤害简称「暴伤」。}                                                                                      \\
	{天赋等级}     & {用连续的数字写下角色天赋等级,包含命之座提供的等级,但不包含特殊因素(例如达达利亚提供的等级)。101010 可以简写为 310。存在超出 10 级的,可以写成 9119 这样的形式。9111 这样存在歧义的,写作 9/1/11 或 9/11/1。如果闲得蛋疼,可以用三个十六进制数字表示,例如 9/10/13 可以写作 9AD。} \\
	\hline
\end{tabu}


\section{配队命名}

\subsection{固定搭配命名法}

部分固定搭配已经众所周知,所以我们承认并沿用这些固定搭配。

\noindent\begin{tabu}{|c|c|X[c]|}
	\hline
	{固定搭配名称} & {简称} & {成员}                                \\
	\hline
	% \endhead
	{国家队}       & {-}    & {班尼特 + 香菱 + 行秋 + 重云}         \\
	{雷神国家队}   & {雷国} & {班尼特 + 香菱 + 行秋 + 雷电将军}     \\
	{万达国际}     & {-}    & {班尼特 + 香菱 + 达达利亚 + 枫原万叶} \\
	{砂糖武装}     & {-}    & {行秋 + 砂糖 + 北斗 + (自由人)} \\
	\hline
\end{tabu}

成员列表排序规则:首先按照元素(火、水、风、雷、草、冰、岩的顺序)分组,然后每个组内按照体型(成男、少男、成女、少女、幼女的顺序)分组,然后每个组内按照所属国家(其他、蒙德、璃月、稻妻、须弥、枫丹、纳塔、至冬的顺序)分组,然后每个组内按照实装版本排序,最后每个组内按照稀有度(五星、四星)排序。


\subsection{角色缩写命名法}

灵活搭配的阵容命名,需要组合阵容中的角色的单字简称。在条件允许的情况下,应当优先选用首选单字简称。但是,为特定修辞效果,允许使用备用单字简称。除非极端特殊需求,不应选用另外的单字简称。

\subsubsection{示例}

\noindent\begin{tabu}{|c|X[c]|}
	\hline
	{缩写命名} & {成员}                                        \\
	\hline
	% \endhead
	{胡行钟阿}     & {胡桃 + 行秋 + 钟离 + 阿贝多}                 \\
	{神罗万心}     & {神里绫华 + 罗莎莉亚 + 枫原万叶 + 珊瑚宫心海} \\
	{莫甘娜温}     & {莫娜 + 甘雨 + 迪奥娜 + 温迪}                 \\
	\hline
\end{tabu}

\subsubsection{查询表}

\begin{longtabu}{|X[2c]|X[c]|X[c]|}
	\hline
	{角色} & {首选单字简称} & {备用单字简称} \\
	\hline
	\endhead
	\hline
	\endfoot
	{迪卢克} & {卢} & {-} \\
	{琴} & {琴} & {-} \\
	{莫娜} & {莫} & {-} \\
	{温迪} & {温} & {-} \\
	{可莉} & {可} & {-} \\
	{阿贝多} & {阿} & {-} \\
	{优菈} & {优} & {-} \\
	\hline
	{安柏} & {安} & {-} \\
	{丽莎} & {丽} & {-} \\
	{凯亚} & {凯} & {-} \\
	{班尼特} & {班} & {点} \\
	{芭芭拉} & {芭} & {-} \\
	{雷泽} & {泽} & {-} \\
	{诺艾尔} & {诺} & {-} \\
	{砂糖} & {砂} & {-} \\
	{迪奥娜} & {娜} & {猫} \\
	{菲谢尔} & {菲} & {皇} \\
	{罗莎莉亚} & {罗} & {修} \\
	\hline
	{刻晴} & {刻} & {-} \\
	{七七} & {七} & {-} \\
	{钟离} & {钟} & {-} \\
	{甘雨} & {甘} & {-} \\
	{魈} & {魈} & {-} \\
	{胡桃} & {胡} & {-} \\
	{申鹤} & {申} & {鹤} \\
	{白术} & {白} & {-} \\
	\hline
	{香菱} & {香} & {-} \\
	{行秋} & {行} & {秋} \\
	{凝光} & {凝} & {-} \\
	{北斗} & {北} & {-} \\
	{烟绯} & {烟} & {-} \\
	{云堇} & {云} & {-} \\
	{重云} & {重} & {-} \\
	{辛焱} & {辛} & {-} \\
	\hline
	{枫原万叶} & {万} & {-} \\
	{神里绫华} & {神} & {-} \\
	{宵宫} & {宵} & {-} \\
	{雷电将军} & {雷} & {影} \\
	{珊瑚宫心海} & {心} & {-} \\
	{荒泷一斗} & {一} & {斗} \\
	{八重神子} & {八} & {-} \\
	{神里绫人} & {人} & {绫} \\
	% {XX} & {XX} & {-} \\
	\hline
	{托马} & {托} & {-} \\
	{五郎} & {五} & {-} \\
	{九条裟罗} & {九} & {裟} \\
	{早柚} & {柚} & {-} \\
	\hline
	{达达利亚} & {达} & {公} \\
\end{longtabu}




\section{武器命名}

除非「特殊昵称」有专门规定,应当遵循「基本称呼」的原则。

\subsection{基本称呼}

全名不超过 3 个字的武器,应当使用其全名称呼。超过 3 个字的,可以使用全名,也可以按照如下规则使用简称。对于属于一个系列的武器,如果与角色一同提及,那么可以仅称呼系列(「天空」),省略具体型号(「之傲」)。

\begin{itemize}
	\item 如果属于「AB 之 CD」格式(例如:雾切之回光),那么简称为 AB。
	\item 如果属于「AB 之 C」格式(例如:破魔之弓),那么简称为 AB。
	\item 如果属于「ABC 之 D」格式(例如:阿莫斯之弓),那么简称为 ABC。
	\item 如果属于「AB 之 X」或「ABC 之 X」格式,但 X 不是武器类型描述(例如:贯虹之槊),那么简称为「之」字之前的字(AB 或 ABC)。
	\item 如果不适用以上任何规则,那么简称为前两个字。
\end{itemize}

\subsection{特殊昵称}

\subsubsection{五星武器}

\begin{tasks}(3)
	\task 狼的末路:狼末
	\task 和璞鸢:鸟枪
	\task 天空之脊:薄荷
	\task 阿莫斯之弓:阿莫斯
	\task 天空之卷:厕纸
	\task 贯虹之槊:贯虹/盾枪
	\task 磐岩结绿:绿剑
	\task 雾切之回光:雾切
\end{tasks}

\subsubsection{四星武器}

\begin{tasks}(3)
	\task 嘟嘟可故事集:嘟嘟可
	\task 衔珠海皇:咸鱼
	\task 喜多院十文字:喜多院
	\task 「渔获」:鱼叉
	\task 试作古华:古华
	\task 辰砂之纺锤:纺锤
\end{tasks}




\section{圣遗物命名}

\subsection{圣遗物套装命名}

\noindent\begin{tabu}{|c|X[c2.5]|X[c]|X[c]|}
	\hline
	{产地} & {套装名称}       & {主要简称} & {备用简称}  \\
	\hline
	% \endhead
	% \hline
	% \endfoot
	{}     & {角斗士的终幕礼} & {角斗}     & {-}         \\
	{}     & {流浪大地的乐团} & {乐团}     & {-}         \\
	\hline
	{蒙德} & {翠绿之影}       & {风}       & {-}         \\
	{}     & {被怜爱的少女}   & {少女}     & {-}         \\
	{}     & {如雷的盛怒}     & {如雷}     & {雷}        \\
	{}     & {平息鸣雷的尊者} & {平雷}     & {-}         \\
	{}     & {沉沦之心}       & {水}       & {-}         \\
	{}     & {冰风迷途的勇士} & {冰}       & {-}         \\
	\hline
	{璃月} & {炽烈的炎之魔女} & {魔女}     & {火}        \\
	{}     & {渡过烈火的贤人} & {渡火}     & {-}         \\
	{}     & {悠古的磐岩}     & {磐岩}     & {岩}        \\
	{}     & {逆飞的流星}     & {逆飞}     & {-}         \\
	{}     & {昔日宗室之仪}   & {宗室}     & {-}         \\
	{}     & {染血的骑士道}   & {骑士}     & {染血}      \\
	{}     & {千岩牢固}       & {千岩}     & {-}         \\
	{}     & {苍白之火}       & {苍白}     & {-}         \\
	{}     & {辰砂往生录}     & {辰砂}     & {流血}      \\
	{}     & {来歆余响}       & {余响}     & {普攻}      \\
	\hline
	{稻妻} & {绝缘之旗印}     & {绝缘}     & {充能}      \\
	{}     & {追忆之注连}     & {追忆}     & {-}         \\
	{}     & {华馆梦醒形骸记} & {华馆}     & {防御}      \\
	{}     & {海染砗磲}       & {海染}     & {蚌埠/毒奶} \\
	\hline
\end{tabu}

\subsection{圣遗物主属性命名}

\begin{longtabu} {|X[c]|X[c]|}
	\hline
	{主属性}         & {简称} \\
	\hline
	\endhead
	\hline
	\endfoot
	{攻击力}         & {攻}   \\
	{防御力}         & {防}   \\
	{生命值}         & {生}   \\
	{治疗加成}         & {治}   \\
	% \hline
	{物理伤害加成}   & {物}   \\
	{火元素伤害加成} & {火}   \\
	{水元素伤害加成} & {水}   \\
	{风元素伤害加成} & {风}   \\
	{雷元素伤害加成} & {雷}   \\
	{草元素伤害加成} & {草}   \\
	{冰元素伤害加成} & {冰}   \\
	{岩元素伤害加成} & {岩}   \\
	\hline
	{元素充能效率}   & {充}   \\
	{元素精通}       & {精}   \\
	\hline
	{暴击率}         & {暴}   \\
	{暴击伤害}       & {暴伤} \\
\end{longtabu}




\section{圣遗物密境命名}

\noindent\begin{tabu}{|c|X[c]|c|c|c|}
	\hline
	{地区} & {密境}         & {圣遗物套装} & {主要名称} & {备用名称} \\
	\hline
	% \endhead
	% \hline
	% \endfoot
	{蒙德} & {铭记之谷}     & {风/少女}    & {风本}     & {-}        \\
	{ }    & {仲夏庭园}     & {如雷/平雷}  & {雷本}     & {-}        \\
	{ }    & {芬德尼尔之顶} & {水/冰}      & {冰本}     & {-}        \\
	\hline
	{璃月} & {无妄引咎密宫} & {魔女/渡火}  & {火本}     & {-}        \\
	{ }    & {华池岩岫}     & {宗室/骑士}  & {宗室本}   & {-}        \\
	{ }    & {孤云凌霄之处} & {磐岩/逆飞}  & {岩本}     & {-}        \\
	{ }    & {山脊守望}     & {千岩/苍白}  & {千岩本}   & {苍白本}   \\
	{ }    & {?}           & {辰砂/余响}  & {-}        & {-}        \\
	\hline
	{稻妻} & {椛染之庭}     & {绝缘/追忆}  & {绝缘本}   & {充能本}   \\
	{ }    & {沉眠之庭}     & {华馆/海染}  & {防御本}   & {-}        \\
	\hline
\end{tabu}



\end{document}
