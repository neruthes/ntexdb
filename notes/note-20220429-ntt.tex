\documentclass[a4paper,11pt]{article}
\usepackage{titling}
\usepackage{enumerate,enumitem,paralist}

% Use letterhead1 library
\input{/home/neruthes/DEV/ntexlibs/lib/letterhead1.tex}








% Fonts
\setmainfont{NewComputerModern}
\setromanfont{NewComputerModern}
\setsansfont{Inter}
\setmonofont{JetBrains Mono NL}
\setCJKmainfont{FandolSong}
\setCJKromanfont{FandolSong}
\setCJKsansfont{Noto Sans CJK SC}
\setCJKmonofont{Noto Sans CJK SC}

% Use alternative geometry
\newgeometry{textwidth=42em,tmargin=45mm,bmargin=33mm}

% % Paragraph style
% \setlength{\parindent}{2em}
% \setlength{\parskip}{4pt}

\newcommand{\Nmaketitle}[1]{%
    \begin{minipage}{\linewidth}%
        \parskip=6pt%
        \center%
        \rmfamily%
        \fontspec[Ligatures=TeX]{Latin Modern Roman}%
        {\LARGE#1}\par%
        {\sffamily\normalsize\theauthor}\par%
        {\sffamily\normalsize\thedate}%
    \end{minipage}\par
    \vspace{15pt}\hrule%
    \vspace{15pt}%
    \renewcommand{\ifootcornerL}[0]{#1}%
}
\newcommand{\Nmaketoc}[0]{\vspace{20pt}\hrule\tableofcontents\vspace{20pt}\hrule}


\renewcommand{\letterfooterfont}[0]{\sffamily\footnotesize}
\renewcommand{\ifootcornerL}[0]{A Note by \theauthor}
\renewcommand{\ifootcornerR}[0]{Page \thepage}

\newcommand{\icode}[1]{\texttt{\footnotesize#1}}


\docsetup{Introduction to Non-Transferable Token (NTT)}{Neruthes}{2022-04-29}
\linespread{1.1}



\begin{document}
% \raggedright

\fulldoctitle












\section*{Abstract}

Non-Transferable Token (NTT) is inspired by NFT, but has nothing to do with the blockchain technology.

A NTT is a Signed Token Grant Certificate (STGC) granted to a Certified Token Owner (CTO) by a Token Issuing Authority (TIA),
who is the owner of a fully qualified domain name (FQDN)
and shall publish its Root Certificate for others to verify issued NTTs.

A NTT cannot be transferred to another owner.



\Nmaketoc












\section{Trust Chain}

The Trust Chain is established on the following levels, ordered from root to ceil.

\begin{compactitem}
	\item OS-wide CA
	\item Any of:
    \item TLS-guarded website
	\item Token Issuing Authority Root Certificate (TIA self-signed)
	\item Token Grant Certificate (Signed by TIA)
\end{compactitem}












\section{Authority Initialization}


\subsection{The Principles}

In modern days, website owners do not necessarily control the private key which safeguards their websites via TLS,
therefore, we shall not require using the TLS certificate of the website for this purpose.
Instead, an independent keypair shall be used.


\subsection{The Procedure}

\subsubsection{SSH Keygen}

A Token Issuing Authority (TIA) must first initialize its Ed25519 keypair using \icode{ssh-keygen}.

The TIA shall create an Ed25519 keypair (for signing purposes), and this keypair is called ``TIA Signing Keypair'' (TIASK).
Suppose that the TIASK is a generic \icode{id\_ed25519} file.

\subsubsection{Authority Keypair Declaration}

The TIA shall write an Authority Keypair Declaration (AKD) which looks like:

\begin{lstlisting}
spec: NTT-Authority-Keypair-Declaration
domain: example.com
key-type: ssh-ed25519
pubkey: AAAAC3NzaC1lZDI1NTE5AAAAIPoPvHErCcKsYF3BrPdfmwgRICsx2XMl1fN9I3Jg1pPA
date: 2022-04-29T23:57:52+00:00
\end{lstlisting}

\subsubsection{Signing Authority Keypair Declaration}

The TIA shall sign the AKD with the TIASK with \icode{ssh-keygen} to make a signature file:

\begin{lstlisting}
ssh-keygen -Y sign -f ~/.ssh/id_ed25519 -n file example.com.akd
\end{lstlisting}

\subsubsection{Publishing Signed Authority Keypair Declaration}

By concatenating the two files, the TIA makes a Signed Authority Keypair Declaration (SAKD), which shall be served properly.
The file should look like:

\begin{lstlisting}
spec: NTT-Authority-Keypair-Declaration
date: 2022-04-29T23:57:52+00:00
domain: example.com
key-type: ssh-ed25519
pubkey: AAAAC3NzaC1lZDI1NTE5AAAAIPoPvHErCcKsYF3BrPdfmwgRICsx2XMl1fN9I3Jg1pPA
-----BEGIN SSH SIGNATURE-----
U1NIU0lHAAAAAQAAADMAAAALc3NoLWVkMjU1MTkAAAAg+g+8cSsJwqxgXcGs91+bCBEgKz
HZcyXV830jcmDWk8AAAAAEZmlsZQAAAAAAAAAGc2hhNTEyAAAAUwAAAAtzc2gtZWQyNTUx
OQAAAEAIpWjhNl6UXd8TFwZ8GwBK4Ntw5BzEHjf92Nyjr21ZKuOmcWDA8m8DLmuQshObgI
Wj9ejDT4Ni1BtxpdGxo2kG
-----END SSH SIGNATURE-----
\end{lstlisting}

The SAKD file should be available at \icode{https://example.com/example.com.sakd}.
It is reasonable that the owner of \icode{example.com} prefers to let \icode{www.example.com} be the canonical domain for its main website.
Therefore, when retrieving the SAKD, 301/302 redirecting shall be allowed as long as the final \icode{200 OK} URL is not out of the TIA domain.












\section{Owner Identification}

The Certified Token Owner (CTO) shall be identified by a SSH public key or an OpenPGP public key, producing a CTOID.


\subsection{SSH Public Key}

If the owner is a SSH public key, it should be identified by the normalized text representation of it (only the first two columns), e.g.:

\begin{lstlisting}
ssh-ed25519 AAAAC3NzaC1lZDI1NTE5AAAAIPoPvHErCcKsYF3BrPdfmwgRICsx2XMl1fN9I3Jg1pPA
\end{lstlisting}

However, for better compatibility, we should slightly adjust its format to produce the CTOID:

\begin{lstlisting}
ssh-ed25519.AAAAC3NzaC1lZDI1NTE5AAAAIPoPvHErCcKsYF3BrPdfmwgRICsx2XMl1fN9I3Jg1pPA
\end{lstlisting}


\subsection{OpenPGP Public Key}

If the owner is an OpenPGP public key, we can only identify it by its 40-digit fingerprint (uppercase, no delimiter), instead of the full public key,
because it contains volatile data such as expiry date and subkeys.
Even though the SHA-1 hash algorithm is miserably vulnerable for collisions.

Example CTOID:

\begin{lstlisting}
CB0ABC7756440D12915E3F25AFB3373F5200DF38
\end{lstlisting}











\section{Designing Token Series}


\subsection{Basic Information}

A NTT must belong to a Token Series.
A Token Series shall contain the following information:

\begin{tabu}{|l|X|}
	\hline
	{Field}             & {Description}                                                                    \\
	\hline
	\icode{tia}         & {The TIA of the series, e.g. \icode{com.example}.}                            \\
	\icode{series-id}   & {A reverse domain name, e.g. \icode{com.example.senator.n68}.}                \\
	\icode{series-name} & {A human-friendly name of the series, e.g. ``Senator of the 68th Congress''.} \\
	\hline
\end{tabu}


\subsection{Human-Readable Information}

More information (e.g. artwork) is necessary for a NTT to make sense in front of humans.
Therefore, the TIA shall host more information of its Token Series in a Token Series Catalog (TSC).

The TSC shall be available at \icode{https://example.com/example.com.tsc.json} as a JSON file.

Example:\pagebreak

\begin{lstlisting}
{
    "spec": "NTT-Token-Series-Catalog",
    "list": [
        {
            "series-id": "com.example.senator.n68",
            "series-name": "Senator of the 68th Congress",
            "artworks": [
                {
                    "type": "square",
                    "url": "https://..."
                },
                {
                    "type": "circle",
                    "url": "https://..."
                }
            ]
        }
    ]
}
\end{lstlisting}

The TSC does not need a digital signature.











\section{Issuing Token}


\subsection{Authority Database}

The TIA shall maintain a database to record its issuing history,
and may optionally publish it at \icode{https://example.com/example.com.tih.json} as a JSON file, where ``tih'' means Token Issuing History.


\subsection{Writing Token Grant Certificate}

The Token Grant Certificate (TGC) shall contain information of the Certified Token Owner (CTO) and the Token Series.

Information of the CTO:

\begin{compactitem}
	\item \icode{name}
	\item \icode{id-type}
	\item \icode{ctoid}
\end{compactitem}

Information of the Token Series:

\begin{compactitem}
	\item \icode{series-id}
	\item \icode{series-name}
\end{compactitem}

In writing, the data fields from the two sources shall be prefixed by their respective source identifiers, i.e. ``cto'' and ``series''.

Example:

\begin{lstlisting}
spec: NTT-Token-Grant-Certificate
date: 2022-04-29T23:57:52+00:00
cto-name: Neruthes
cto-id-type: ssh
cto-ctoid: ssh-ed25519.AAAAC3NzaC1lZDI1NTE5AAAAIPoPvHErCcKsYF3BrPdfmwgRICsx2XMl1fN9I3Jg1pPA
series-series-id: com.example.senator.n68
series-series-name: Senator of the 68th Congress
\end{lstlisting}


\subsection{Signing Token Grant Certificate}

The certificate body shall be signed (in the same way as we make the SAKD) to produce the Signed Token Grant Certificate (STGC).

\begin{lstlisting}
spec: NTT-Token-Grant-Certificate
date: 2022-04-29T23:57:52+00:00
cto-name: Neruthes
cto-id-type: ssh
cto-ctoid: ssh-ed25519.AAAAC3NzaC1lZDI1NTE5AAAAIPoPvHErCcKsYF3BrPdfmwgRICsx2XMl1fN9I3Jg1pPA
series-series-id: com.example.senator.n68
series-series-name: Senator of the 68th Congress
-----BEGIN SSH SIGNATURE-----
U1NIU0lHAAAAAQAAADMAAAALc3NoLWVkMjU1MTkAAAAg+g+8cSsJwqxgXcGs91+bCBEgKz
HZcyXV830jcmDWk8AAAAAEZmlsZQAAAAAAAAAGc2hhNTEyAAAAUwAAAAtzc2gtZWQyNTUx
OQAAAEC3qyZ8Eiye0LtLh8csJhKRbrvB6ESpFjHyrACXSws9KbS8GBmaCNzASJj+LcfJTj
b1bjPSO/cmmNaDePNWrOAH
-----END SSH SIGNATURE-----
\end{lstlisting}


\subsection{Distributing Signed Token Grant Certificate}

The STGC shall be sent to the CTO itself.

In addition to its own TIH file, the TIA may optionally publish the STGC to some Token Ownership Indexing Servers for a better indexing.

A Token Ownership Indexing Servers (TOIS) is a server which maintains mapping relations from CTOID to STGC.









\section{Presenting Token}

Ideally, a User Agent (UA), e.g. a mobile app, shall be responsible for managing the list of favorable STGCs.
When necessary, the UA shall present the NTT by displaying the STGC.

Methods of presentation include (but with no limitation to) NFC and QR.












\section{Validating Token}

% \subsection{Basic Steps}
The validation procedure consists of 5 steps:

\begin{compactenum}
    \item Obtain STGC
    \item Obtain SAKD
    \item Verify SAKD
    \item (Optional) Obtain OKL
    \item Verify STGC
\end{compactenum}











\section{Extended Features}


\subsection{Authority Keypair Retiring}

The are occasions that the TIA wants to retire an old keypair and migrate to a new keypair, such as a data loss incident.
In such cases, the TIA may generate a new keypair, leaving previously signed NTTs unverifiable.

For this matter, the AKD may include a new field \icode{oldkeys},
which points to a HTTPS URL where the Old Keys List (OKL) file shall be published.
The URL shall contain a list of SSH public keys.
The file format shall conform to a \icode{known\_hosts} file.


\subsection{Owner-Defined Whitelist for Authorities}

A attacker may spam a Token Ownership Indexing Server (TOIS) by issuing massive STGCs to a particular CTOID.

To prevent this, a TOIS may support this feature, which allows the CTO to defined a whitelist of allowed TIAs,
so that the TOIS may safely discard the incoming STGCs for the CTO from unlisted TIAs with a variety of possible policies,
e.g. remove the STGC within 7 days unless the CTO whitelists the TIA.

A Owner-Defined Whitelist for Authorities (ODWA) shall look like this:

\begin{lstlisting}
spec: NTT-Owner-Defined-Whitelist-For-Authorities
date: 2022-04-29T23:57:52+00:00
cto-name: Neruthes
cto-id-type: ssh
cto-ctoid: ssh-ed25519.AAAAC3NzaC1lZDI1NTE5AAAAIPoPvHErCcKsYF3BrPdfmwgRICsx2XMl1fN9I3Jg1pPA
tia-whitelist: example.com,neruthes.xyz
\end{lstlisting}

The ODWA shall be signed similarly with \icode{ssh-keygen}, or signed with \icode{gpg --clear-sign}.

The \icode{tia-whitelist} shall be a comma-delimited string for domains.










\section{Extra Notes}

\begin{compactitem}
	\item All text files shall be encoded in UTF-8, shall use LF as line terminator, and shall end with a LF.
\end{compactitem}















\end{document}
