%!TEX TS-program = xelatex
%!TEX encoding = UTF-8 Unicode

\documentclass[a4paper,12pt]{article}

\usepackage[a4paper,hmargin=20mm,tmargin=20mm,bmargin=20mm]{geometry}
\usepackage[dvipsnames]{xcolor}
\usepackage{calc}
\usepackage{amsmath,xltxtra,xunicode}
\usepackage{titlesec}
\usepackage{fontspec}
\usepackage{wasysym,oz,pxfonts,txfonts}


\usepackage[PunctStyle=plain,RubberPunctSkip=false]{xeCJK}
\XeTeXlinebreaklocale "zh" 
\XeTeXlinebreakskip = 0pt plus 1pt 

% =========================================
\usepackage{everypage}
\usepackage{listings}
\usepackage{paralist}
\usepackage{enumerate}
\usepackage{enumitem}
\usepackage{tocloft}
\usepackage{longtable}
\usepackage{tabu}
\usepackage{makecell}
\usepackage{qrcode}
\usepackage{graphicx}
\usepackage{tikz}
\usepackage{eso-pic}
\graphicspath{ {/home/neruthes/DEV/ntexlibs/pic} }
\usepackage{fontawesome5}
\usepackage{multirow}
\usepackage{multicol}
\usepackage{ragged2e}
\usepackage{tcolorbox}
\setmainfont{JetBrains Mono}
\setromanfont{Charter}
\setsansfont{SF Pro Text}
\setmonofont{JetBrains Mono}
\setCJKmainfont{Noto Sans CJK SC}
\setCJKromanfont{Noto Serif CJK SC}
\setCJKsansfont{Noto Sans CJK SC}
\setCJKmonofont{Noto Sans CJK SC}

\newcommand{\kaifamily}[0]{\CJKfontspec{FandolKai}}

\title{Fapiao}
\author{Neruthes}
\date{2021-09-17}
\pagestyle{plain}
% =========================================

% =========================================
% START CORPORATION CONFIG
\setlength{\parindent}{0pt}
\setlength{\parskip}{1pt}
\setlength{\baselineskip}{12pt}
% END CORPORATION CONFIG
% =========================================

% =========================================
% START COMMON STYLESHEET
% Table
\renewcommand\theadalign{l}
\renewcommand\theadfont{\sffamily\bfseries}
% Font Size

% END COMMON STYLESHEET
% =========================================

\colorlet{FapiaoYellow}{orange!66!black}
\tabulinesep=5pt

\begin{document}
    \pagestyle{empty}
    \sffamily



    \begin{titlepage}
        \hspace{1pt}
        \vspace{100pt}

        \center
        \bfseries
        \Huge
        原神的召唤\\
        \huge
        Call of Genshin

        \vspace{10pt}
        \large\mdseries
        桌面游戏扮演游戏规则书

        \vfill
        \normalsize\mdseries
        Neruthes\\
        2021-10-12\\
        内部预览草稿
        \vspace{100pt}
    \end{titlepage}

    \section{前言}

    本文旨在介绍基于原神世界观和 COC 玩法的 TRPG 框架设计。

    KP 需要学习《高等元素论》。

    \section{车卡}

    可以采用购点法或 roll 点法。数值基于原神。骰子规则基于 COC 第七版。

    \subsection{战斗类点数}

    战斗类点数如下:

    \begin{tabu} {|llXl|}
        \hline
        字段 & 基础值 & 介绍 & 每点成长 \\
        \hline
        攻击力 & 300 & 用于计算输出的伤害 & 10 \\
        防御力 & 300 & 用于计算受到的伤害 & 10 \\
        生命值 & 10000 & 生命值 & 250 \\
        元素充能效率 & 100\% & 用于计算元素充能的倍率 & 5\% \\
        元素精通 & 0 & 用于计算元素反应伤害加成和元素能量存储上限 & 5 \\
        \hline
    \end{tabu}

    \subsection{非战斗类点数}

    沿用参考 COC 第七版。

    \section{战斗}

    游戏采用回合制战斗。

    在每个回合,每个角色可以做出 1 个基础动作。部分基础动作允许额外动作。额外动作分为前摇和后摇。部分基础动作需要占用连续 2 个回合。

    \subsection{基础动作}

    \subsubsection{普通攻击}

    根据武器类型选择倍率:

    \begin{tabu} {|X|X|X|X|X|}
        \hline
        单手剑 & 双手剑 & 长柄武器 & 法器 & 弓箭 \\
        \hline
        50\% & 75\% & 50\% & 50\% & 25\% \\
        \hline
    \end{tabu}


    备注:

    \begin{itemize}
        \item 双手剑角色每 3 次普通攻击后需要休息 1 回合恢复体力。休息时不能使用普通攻击、下落攻击、元素战技。可以使用元素爆发,但休息回合顺延至下一回合。
        \item 弓箭角色的普通攻击可以变为蓄力攻击,造成 50\% 倍率的伤害。
        \item 弓箭角色在一次普通攻击(非蓄力)后,下一回合可以连续两次普通攻击。
    \end{itemize}


    \subsubsection{下落攻击}

    在普通攻击的基础上,额外获得 20\% 的倍率。但是,根据下落的高度,需要进行坠落受伤判定(1d10 到 1d100 不等的生命值比例扣减)。

    \subsubsection{元素战技}

    在车卡时决定元素战技的类型。类型可以是伤害、召唤(嘲讽)、召唤(输出)、附魔。

    \begin{itemize}
        \item 伤害:\\
                按照普通攻击的数值,对敌人造成自身神之眼的属性的伤害。\\
                回复 10 点元素能量。回复量享受元素充能效率的加成。\\
                冷却 1 回合。
        \item 召唤(嘲讽):\\
                创建召唤物,在连续 2 个回合对敌人造成嘲讽效果;敌人攻击动作发起前需要与召唤者意志对抗,失败时必然攻击召唤物。\\
                回复 10 点元素能量。回复量享受元素充能效率的加成。\\
                冷却 2 回合。
        \item 召唤(输出):\\
                创建召唤物,在连续 2 个回合对敌人造成等于自身普通攻击数值的伤害;敌人攻击动作发起前需要与召唤者意志对抗,失败时必然攻击召唤物。\\
                回复 10 点元素能量。回复量享受元素充能效率的加成。\\
                冷却 2 回合。
        \item 附魔:\\
                附魔自己的武器,使接下来连续 2 回合的普通攻击、下落攻击可以造成元素伤害。\\
                回复 10 点元素能量。回复量享受元素充能效率的加成。\\
                冷却 3 回合。
    \end{itemize}

    \subsubsection{元素爆发}
    \subsection{额外动作}
    \subsubsection{元素附着前摇}

    \subsection{伤害结算规则}

    \subsubsection{结算顺序}

    \begin{itemize}
        \item 攻击内含伤害(攻击力乘区,倍率乘区,增幅反应乘区)
        \item 剧变反应伤害(扩散,超载,冻结,燃烧,碎冰,超导,结晶)
        \item 扩散反应造成的进一步剧变反应伤害
    \end{itemize}
    
    \subsubsection{计算规则}

    123123123123123

    \subsubsection{备注}

    \begin{itemize}
        \item 水雷共存时,进入的元素首先与雷元素反应;耗尽雷元素后仍然剩余至少 0.5 元素量,则会继续与水元素反应。
        \item 
    \end{itemize}




\end{document}
