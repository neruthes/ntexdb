%!TEX TS-program = xelatex
%!TEX encoding = UTF-8 Unicode

\documentclass[a4paper,11pt]{article}
\input{/home/neruthes/DEV/ntexlibs/lib/letterhead1.tex}


\usepackage{tabu}
\usepackage{tocloft}
\usepackage[bottom]{footmisc}
\usepackage[hidelinks]{hyperref}
\usepackage{paralist,enumitem}
\usepackage{indentfirst}

% \setsansfont{Inter}
\setsansfont{Noto Sans CJK SC}
\setromanfont{Noto Serif CJK SC}
\setCJKromanfont{Noto Serif CJK SC}





\newcommand{\smallnote}[1]{{\sffamily\footnotesize#1}}


% Paper Size
\setlength{\textwidth}{\paperwidth-50mm}


\usepagenumfoot
\linespread{1.25}


\begin{document}
\rmfamily

\recipientaddress{
    \fontspec{Alegreya Sans}
    \parskip=10pt
    {\bfseries\Large\scshape Request For Comments}\\
    {\small Written for AOSC (Anthon Open Source Community)}

    \ttfamily
    \tabcolsep=0pt
    \begin{tabu} {lll}
        Date:    & {}             & 2022-03-15 \\
        Version: & {}             & 0.1.1 \\
        Author:  & {}             & Neruthes   \\
        {}       & {\hspace{2em}} & {}         \\
    \end{tabu}
}






{\sffamily\huge\bfseries [RFC] OSPP 项目管理指导意见}
\vspace{12pt}

{\fontsize{10pt}{10pt}\selectfont\tableofcontents}

\vspace{2em}
\hrule
\vspace{1.5em}







\parindent=2em
\setdefaultleftmargin{2em}{1em}{1em}{1em}{1em}{1em}





% \section*{背景介绍}

自 2020 年起,本社区积极参与每年的 OSPP 活动。在相关活动中,本社区任命导师,OSPP 活动主办方派遣学生,导师与学生合作完成一些项目工程。这对本社区、OSPP、导师本人、学生本人都是有利的。

不幸的是,2021 年部分项目的执行工作出现了项目管理灾难。作者对相关史实开展过有限的调研,并希望协助本社区在 2022 年度避免重蹈覆辙。

特此撰写本文,分析工作方针的改进方向,以供本社区的老伙计们交流讨论。本文主要探讨软件开发类型的项目,但在执行其他类型的项目时,亦可加以有限的参考。










\section{经济问题}

// TODO










\section{事故复盘}

基于在 2021 年度出现的事故,结合生产生活中的一般常识和经验,可以总结如下风险点:

\begin{compactitem}
	\item \textbf{需求管理}\\产品需求是所有严肃的软件项目的首要问题。
	\item \textbf{进度管理}\\并不会有无限的时间用来开发,所以必须重视进度。
	\item \textbf{技术选型}\\该项目的技术选型使用 Python,但导师并不擅长 Python。技术选型需要考虑导师的指导、救场能力。
\end{compactitem}









\section{开工前的准备}

\subsection{时间线}

\begin{compactitem}
    \item 社区内部确定需求
    \item 社区内部确定边界
\end{compactitem}


\subsection{确定的需求}

一个软件产品应当有确定的需求。为此,在开工前,需要对以下问题做出明确的规定:

\begin{compactitem}
	\item 由谁使用?用在什么场景?服务于使用者的什么目标?
	\item 有哪些核心功能?有哪些次要功能?
	\item 在功能目标之外,有哪些其他目标?\smallnote{(文档、manpage、教程……)}
\end{compactitem}

在需求商议阶段,导师应当主导相关讨论,本社区内对此项目关心的老伙计们可以发表意见。学生可以不参与本阶段工作。

\subsection{清晰的边界}

在需求确定后,我们需要划定边界。为此,在开工前,需要对以下问题做出明确的规定:

\begin{compactitem}
	\item 我们要求它有多久的生命周期?\smallnote{(一年、五年、十年……)}
	\item 我们允许它依赖怎样的硬件条件?\smallnote{(amd64、aarch64、ppc64el……)}
	\item 我们允许它依赖怎样的软件条件?\smallnote{(gcc、rustc、llvm……)}
\end{compactitem}

学生可以不参与本阶段工作。

\subsection{技术选型}

合理的技术选型,是软件产品顺利开发、持续维护的前提条件。为此,我们需要考虑以下问题:

\begin{compactitem}
	\item 导师本人擅长哪些技术栈?
	\item 社区里的其他活跃开发者能够协助维护哪些技术栈的源代码?
	\item 学生本人擅长哪些技术栈?
\end{compactitem}

学生应当参与本阶段工作。

\subsection{前期文档}

在实际开工之前,我们需要形成如下文档:

\begin{compactitem}
	\item 产品需求文档\smallnote{(列明需求)}
	\item 软件设计规范文档\smallnote{(规定工作边界,设计具体行为)}
	\item 技术选型文档\smallnote{(指明技术栈和依赖)}
\end{compactitem}

前期文档由导师主导撰写。此外,项目管理志愿者\footnote{人选包括:Neruthes}可以协助维护文档管理的基础设施,可以在导师的邀请下参与文档撰写。前期文档的内容应当反映导师和本社区内对此项目关心的老伙计们的一致意见。学生可以不参与前期文档的前置工作,只等待前期文档成品,但技术选型问题例外。

可以参考作者在 \texttt{aoscdk-rs} 项目上提供的文档协助\footnote{\href{https://neruthesgithubdistweb.vercel.app/miscdoc/deploykit/01-prd.pdf}{https://neruthesgithubdistweb.vercel.app/miscdoc/deploykit/01-prd.pdf}}。









\section{开工后的日常}

\subsection{目标细化}

我们需要分析,这个软件产品是否可以分阶段 deliver。如果可以,那么我们应当将整个开发计划拆分为多个 phase(阶段)。每个 phase 的定义,包含以下要素:

\begin{compactitem}
    \item 需要验收什么成果
    \item 认为合理的工期是多久
\end{compactitem}

在一个 phase 得到验收之前,下一个 phase 不应开启。每个 phase 的验收标准,导师应当咨询社区其他老伙计的意见。

\subsection{进度追踪}

对于每个 phase,还可以细分为多个 milestone(里程碑),每个 milestone 的预计时间应为 1 周或 2 周。如果一个 phase 足够小,它可以被视为有且只有 1 个 milestone。

一个 milestone 应当包含确定数量的 issue。在 GitHub 上,可以使用 milestone 功能来实现 milestone。导师应当协助学生根据前期文当的规定设置每个 issue。

导师和学生应当约定每周至少 1 次在固定时间正式沟通 milestone 的执行进度。与此相关的一切沟通,应当发生在此项目专门的 Telegram 群内,允许本社区内对此项目关心的老伙计们观摩、提问、发表意见、提供帮助。











\section{异常处理}

\subsection{质量缺陷}

本社区不是商业公司,对软件开发项目的时间要求是软性的。在相对合理的范围内,优先保障软件质量。软件质量保障措施(白盒测试、黑盒测试、单元测试),都需要做起来。

\subsection{交付延期}

如果一个 milestone 未能在预期的时间完成,那么可以允许延期(以请假的形式),但导师必须调查清楚延期的原因,并向本社区内对此项目关心的老伙计们提交解释报告。
















\section{其他问题}

\subsection{后期文档}

在最终验收时,我们需要在上文所述前期文档之外,撰写如下后期文档:

\begin{compactitem}
	\item 用户手册\smallnote{(指导用户该如何使用,作为 manpage 或 wiki)}
	\item 开发者手册\smallnote{(超出技术选型文档的范畴但对将来的开发者必要的信息)}
\end{compactitem}














\end{document}
