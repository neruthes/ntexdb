\documentclass[a4paper,11pt]{article}
\documentclass[a4paper,11pt]{article}
\usepackage{titling}
\usepackage{enumerate,enumitem,paralist}

% Use letterhead1 library
\input{/home/neruthes/DEV/ntexlibs/lib/letterhead1.tex}








% Fonts
\setmainfont{NewComputerModern}
\setromanfont{NewComputerModern}
\setsansfont{Inter}
\setmonofont{JetBrains Mono NL}
\setCJKmainfont{FandolSong}
\setCJKromanfont{FandolSong}
\setCJKsansfont{Noto Sans CJK SC}
\setCJKmonofont{Noto Sans CJK SC}

% Use alternative geometry
\newgeometry{textwidth=42em,tmargin=45mm,bmargin=33mm}

% % Paragraph style
% \setlength{\parindent}{2em}
% \setlength{\parskip}{4pt}

\newcommand{\Nmaketitle}[1]{%
    \begin{minipage}{\linewidth}%
        \parskip=6pt%
        \center%
        \rmfamily%
        \fontspec[Ligatures=TeX]{Latin Modern Roman}%
        {\LARGE#1}\par%
        {\sffamily\normalsize\theauthor}\par%
        {\sffamily\normalsize\thedate}%
    \end{minipage}\par
    \vspace{15pt}\hrule%
    \vspace{15pt}%
    \renewcommand{\ifootcornerL}[0]{#1}%
}
\newcommand{\Nmaketoc}[0]{\vspace{20pt}\hrule\tableofcontents\vspace{20pt}\hrule}


\renewcommand{\letterfooterfont}[0]{\sffamily\footnotesize}
\renewcommand{\ifootcornerL}[0]{A Note by \theauthor}
\renewcommand{\ifootcornerR}[0]{Page \thepage}

\newcommand{\icode}[1]{\texttt{\footnotesize#1}}


\fontsizepteleven
\docsetup{钟离MBTI类型划分问题的主观分析}{Neruthes}{2022-06-04}
\linespread{1.1}







\begin{document}

\fulldoctitle


\section*{摘要}

作者本人属于INTP类型。本文从自身的主观经验和对其他类型人士的印象出发,讨论钟离的人格特色与自身的异同,探讨钟离属于INTP类型的可能性。

本文不探讨MBTI可能存在的缺陷和心理学界对它的批评意见。

虽然钟离只是摩拉克斯卸下岩王帝君的权柄后行走于人间的马甲,钟离也不能简单地等同于摩拉克斯,
但是讨论钟离的人格类型(在钟离这个马甲的范围内讨论摩拉克斯的人格类型)仍然是有意义的。
虽然摩拉克斯不是人类,未必拥有经典意义上的人格,但游戏开发团队是人类,会以钟离为纽带以人格化的方式描绘摩拉克斯的形象。











\section{研究方法}

MBTI类型分析,最重要的是,在8个功能中找出4个主导功能,并正确地排序。
在用语层面我们约定,Thinking是一种功能,Ti是一个功能。
本文将会在Thinking-Feeling维度和Intuition-Sensation维度分别讨论组内两种功能的强弱关系,并在每种功能的范围内讨论内倾、外倾两个功能的强弱关系。

找出每个维度内的首要功能后,我们将探讨它们两者的强弱顺序,以得出完成的四大功能顺序列表,进而得出人格类型描述符。










\section{Thinking vs Feeling}

「以普遍理性而论」这句经典台词,虽然运用得有些强行,但我们能够感知到,编剧希望强调钟离理性的一面,相比于感性的一面。
并且,在诸多台词的字里行间,我们确实感受到了,钟离的T比F更强。

\subsection{Ti vs Te}

作者本人的Ti强度十分离谱,并且不熟悉Te的情况,本文难以具体讨论Thinking功能。
但是作者发现能与钟离对上某些「电波」,所以姑且认定钟离的Thinking功能是Ti大于Te。

「摩拉天然是货币,但货币天然不是摩拉」这句台词,有强烈的Ti色彩——虽然作者也不懂为什么不是Te色彩。

\subsection{Fi vs Fe}

我们在一些剧情中观察到,钟离不擅长对外表达自己内心的情感,但对他人比较宽容。这提示我们,钟离的Fe强于Fi。
作者的Fe是第四功能,主导共情、接纳;Fi是第八功能,主导憎恨、排斥。

具体事例包括:

\begin{compactitem}
	\item 委托旅行者为魈配送药品
	\item 明知宛烟和克列门特并非诚心考古,也不排斥陪他们一程
\end{compactitem}














\section{Intuition vs Sensation}

\subsection{Ni vs Ne}

\subsection{Si vs Se}

在钟离传说任务《古闻之章》第一幕《盐花》中,钟离认定,断开的剑应当算作两件物品。
这是一个有代表性的案例。这个细节提示我们注意,钟离以契约为名义,十分重视形式和程序,堪称偏执。

结合作者本人的情况,这种偏执很可能是Ti-Si主导的。

Se是作者的第七功能,作者完全不能理解Se是一种怎样的功能,所以本文无法从反方面探讨Sensation维度。




\end{document}
