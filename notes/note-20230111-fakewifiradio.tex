\documentclass[a4paper,11pt]{article}
\documentclass[a4paper,11pt]{article}
\usepackage{titling}
\usepackage{enumerate,enumitem,paralist}

% Use letterhead1 library
\input{/home/neruthes/DEV/ntexlibs/lib/letterhead1.tex}








% Fonts
\setmainfont{NewComputerModern}
\setromanfont{NewComputerModern}
\setsansfont{Inter}
\setmonofont{JetBrains Mono NL}
\setCJKmainfont{FandolSong}
\setCJKromanfont{FandolSong}
\setCJKsansfont{Noto Sans CJK SC}
\setCJKmonofont{Noto Sans CJK SC}

% Use alternative geometry
\newgeometry{textwidth=42em,tmargin=45mm,bmargin=33mm}

% % Paragraph style
% \setlength{\parindent}{2em}
% \setlength{\parskip}{4pt}

\newcommand{\Nmaketitle}[1]{%
    \begin{minipage}{\linewidth}%
        \parskip=6pt%
        \center%
        \rmfamily%
        \fontspec[Ligatures=TeX]{Latin Modern Roman}%
        {\LARGE#1}\par%
        {\sffamily\normalsize\theauthor}\par%
        {\sffamily\normalsize\thedate}%
    \end{minipage}\par
    \vspace{15pt}\hrule%
    \vspace{15pt}%
    \renewcommand{\ifootcornerL}[0]{#1}%
}
\newcommand{\Nmaketoc}[0]{\vspace{20pt}\hrule\tableofcontents\vspace{20pt}\hrule}


\renewcommand{\letterfooterfont}[0]{\sffamily\footnotesize}
\renewcommand{\ifootcornerL}[0]{A Note by \theauthor}
\renewcommand{\ifootcornerR}[0]{Page \thepage}

\newcommand{\icode}[1]{\texttt{\footnotesize#1}}


\fontsizepteleven
% \fontsizeptwelve
\docsetup{WiRadio: Protocol Suite for Radio Broadcasting over Wi-Fi Frequencies}{Neruthes}{2023-01-11}
% \linespread{1.1}







\begin{document}
% \raggedright

\fulldoctitle


\section*{Abstract}

I intend to design a hybrid protocol suite, named \textbf{WiRadio}, having both digital and analog features,
for the transmission of realtime audio over safe civilian frequencies such as 2.4 GHz and 5 GHz,
which are commonly used by the IEEE 802.11 protocol suite and Bluetooth across major countries.

The WiRadio protocol suite concerns the transportation of analog audio signals and
a Zeroconf-like service discovery mechanism.

\tableofcontents
\clearpage


\section{Frequencies}

\subsection{Options for Frequencies}

The idea of WiRadio is to conceal the existence of radio stations within the jungle of Wi-Fi signals.
Therefore, only the frequencies used by the IEEE 802.11 protocol suite may be our options.
Particularly, only IEEE 802.11a/b/g/n/ac/ax frequencies may be used.
These include 2.4 GHz frequencies and 5 GHz frequencies:

\begin{itemize}
	\item 2.4 GHz
	      \begin{itemize}
		      \item Channel 1-11 (2412-2462 MHz)
	      \end{itemize}
	\item 5 GHz
	      \begin{itemize}
		      \item Channel 32-68 (5160-5340 MHz)
		      \item Channel 149-161 (5745-5805 MHz)
		      \item Channel 165 (5825 MHz)
	      \end{itemize}
\end{itemize}

\subsection{Dynamic Frequency Selection}

Dynamic Frequency Selection (DFS) is an important feature of modern Wi-Fi communications.
It can be helpful for WiRadio.

WiRadio shall inherit the existing DFS mechanism in IEEE 802.11ax,
whose algorithm and implementations are beyond my knowledge for the moment of this note being written.



\section{Analog Audio Transmission}

WiRadio shall inherit the existing encoding mechanism of FM radio,
which is also beyond my knowledge for the moment of this note being written.

Someone who accidentally turns his FM radio receiver to a specific frequency
shall be able to listen to the audio inside.



\section{Digital Service Discovery Mechanism}

\subsection{Faking Wi-Fi Beacon Frames}

WiRadio shall inherit the existing method of broadcasting beacon frames in IEEE 802.11.
The contents of the fake Wi-Fi beacon frames do not conform to the format specified in IEEE 802.11.
However, the fake Wi-Fi beacon frames shall start with 80 bits which make them to look like real Wi-Fi beacon frames.
As a result, in the eyes of generic Wi-Fi chips, they are receiving malformed Wi-Fi beacon frames.

A WiRadio beacon frame has a dynamic magic header at the initial 10 bytes, which are to be ignored, since they are only a coverup.
The real data starts at the 11th byte.

Here is the structure of the data in a WiRadio beacon frame:

\begin{tabu}{lllX}
	\toprule
	Index & Field Name          & Length   & Description                                                                                                                                                                                                                                                                                                               \\
	\midrule
	0     & Magic Header        & 4B       & {Magic header. Should be 0x19911225.}                                                                                                                                                                                                                                                                                     \\
	1     & Protocol Version    & 1B       & {UInt8 number, for the version of WiRadio protocol being used now. Should be 0x00.}                                                                                                                                                                                                                                       \\
	2     & Timestamp           & 8B       & {UInt64 number, for the UNIX timestamp in milliseconds.}                                                                                                                                                                                                                                                                  \\
	3     & Seed                & 1B       & {UInt8 number, a random number to avoiding collision.}                                                                                                                                                                                                                                                                    \\
	4     & Page                & 1B       & {UInt8 number, to indicate which page this frame belongs to the macro-frame. All pages of a macro-frame must share the same timestamp and the same seed. This is useful for situations where the WiRadio beacon frame becomes too long to be contained in a legitimate-looking fake Wi-Fi beacon frame. Starting from 0.} \\
	5     & Station Pubkey Type & 1B       & {0x00: Ed25519 (with OpenSSH default parameters). Other types may be supported in future.}                                                                                                                                                                                                                                \\
	6     & Station Pubkey      & Variable & {The actual length depends on `Station Pubkey Type'. 0x00: 32 bytes.}                                                                                                                                                                                                                                                     \\
	7     & Station Name Length & 1B       & {UInt8 number, for the length of Field `Station Name'.}                                                                                                                                                                                                                                                                   \\
	8     & Station Name        & Variable & {A string encoded in UTF-8.}                                                                                                                                                                                                                                                                                              \\
	9     & Current Frequency   & 2B       & {UInt16 number, for the current frequency in MHz.}                                                                                                                                                                                                                                                                        \\
	10    & Signature           & Variable & {The predictable-length digital signature for all bytes above, using the hash+signature algorithms specified in `Station Pubkey Type'.}                                                                                                                                                                                   \\
	11    & Hash                & 6B       & {Get the initial 6 bytes from the SHA-256 hash of all bytes above.}                                                                                                                                                                                                                                                       \\
	\bottomrule
\end{tabu}

Note that when a multi-page macro-frame is being broadcasted, a header of Field 0-3 shall be the common header in every page.
Any page must not be longer than 240 bytes.
If the whole macro-frame exceeds the limit, paging is necessary.

% Wi-Fi beacon frame: 252 bytes
% Our usable: 240 bytes

\subsection{Requirement for User Agents}

In an audience ever listens to the radio station, the user agent shall remember the station pubkey,
so that the audience may continue the listening when the station jumps to another frequency using DFS.




\end{document}
