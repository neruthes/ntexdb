%!TEX TS-program = xelatex
%!TEX encoding = UTF-8 Unicode

\documentclass[a4paper,11pt]{article}


\newcommand{\recipientaddress}{
    \begin{minipage}{37.99mm}
        \begin{FlushRight}
            % \raggedright
            \sffamily
            \mdseries
            \footnotesize
            \setlength{\baselineskip}{14pt}
            \setlength{\parskip}{0pt}

            % 总务司\par
            % 璃月港

        \end{FlushRight}
    \end{minipage}
}

% \input{/home/neruthes/DEV/ntexlibs/lib/letterhead1.tex}
\input{/home/neruthes/DEV/ntexlibs/lib/letterhead2.tex}

% \usepackage{showframe}
\usepackage{lipsum}

\begin{document}
    \pagestyle{plain}
    \rmfamily\normalsize
    \setlength{\parskip}{7pt}
    \setlength{\baselineskip}{15pt}
    \raggedright
    \raggedbottom

    \normalsize

    战争已经开始了,是上一场战争的延续。\linebreak
    众神为欲望的轮廓镀上七种光辉,以此昭示,它们的权柄可被企及。\linebreak
    而现世的基底埋藏着阴燃的残骸。那是对僭越者的警示。\linebreak
    「高天之上的神座,从来不是为你预留的位置。」\linebreak
    但僭越之人啊,不要就此驻足。谁都不能隔岸观火。\linebreak
    看吧——

    守护自由城邦千年的巨龙,终于对自由产生了迷茫。\linebreak
    被「自由」之神命令的自由,还能称之为自由吗?

    众目睽睽之下,「契约」之神遭人谋杀。\linebreak
    在最后的时刻,他将签订终结一切契约的契约。

    将军长生不灭,幕府锁国之期亦无尽头。\linebreak
    追求「永恒」之神,在世人眼中见到了怎样的永恒?

    智慧是「智慧」之神的敌人,知识是无知之海表面漂浮的诱饵。\linebreak
    学城的学者正在催生愚行,而神的智慧对此并无意见。

    「正义」之神热爱法庭上的一切闹剧,甚至渴求审判诸神。\linebreak
    但她非常明白,唯有「天理」不可与之为敌。

    战争的规则刻写在生物体内,败者成为战火的余烬,而胜者重燃。\linebreak
    「战争」之神将这秘密告知旅人,因为她有这样做的理由。

    她是人再也不会去爱的神,她是再也不会去爱人的神。\linebreak
    人跟随她的原因,是相信她终有一日能对「天理」举起叛旗。

    在无始无终的永恒里,人类将度过安然无梦的一生。\linebreak
    但在神明视线的死角下,仍有人想要做梦。\linebreak
    人类有人类的底蕴,我们不是滤清「神选者」后剩余的残渣。\linebreak
    从世界之外,我们取得否定世界的力量。

    现在,踏入天地之人,你的旅途已经告终,但你仍未越过最后的门扉。\linebreak
    若你理解旅途的意义,就上前来,击败我,命令我让路,向我证明你比我更适合拯救她。\linebreak
    然后,就去重新纺织所有的命运吧。

    我的记忆已经磨损了太多,但我总还记得,她也喜欢这些花。

    % \lipsum[1]

    % \lipsum[2]



\end{document}