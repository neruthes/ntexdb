\documentclass[a4paper,11pt]{article}
\documentclass[a4paper,11pt]{article}
\usepackage{titling}
\usepackage{enumerate,enumitem,paralist}

% Use letterhead1 library
\input{/home/neruthes/DEV/ntexlibs/lib/letterhead1.tex}








% Fonts
\setmainfont{NewComputerModern}
\setromanfont{NewComputerModern}
\setsansfont{Inter}
\setmonofont{JetBrains Mono NL}
\setCJKmainfont{FandolSong}
\setCJKromanfont{FandolSong}
\setCJKsansfont{Noto Sans CJK SC}
\setCJKmonofont{Noto Sans CJK SC}

% Use alternative geometry
\newgeometry{textwidth=42em,tmargin=45mm,bmargin=33mm}

% % Paragraph style
% \setlength{\parindent}{2em}
% \setlength{\parskip}{4pt}

\newcommand{\Nmaketitle}[1]{%
    \begin{minipage}{\linewidth}%
        \parskip=6pt%
        \center%
        \rmfamily%
        \fontspec[Ligatures=TeX]{Latin Modern Roman}%
        {\LARGE#1}\par%
        {\sffamily\normalsize\theauthor}\par%
        {\sffamily\normalsize\thedate}%
    \end{minipage}\par
    \vspace{15pt}\hrule%
    \vspace{15pt}%
    \renewcommand{\ifootcornerL}[0]{#1}%
}
\newcommand{\Nmaketoc}[0]{\vspace{20pt}\hrule\tableofcontents\vspace{20pt}\hrule}


\renewcommand{\letterfooterfont}[0]{\sffamily\footnotesize}
\renewcommand{\ifootcornerL}[0]{A Note by \theauthor}
\renewcommand{\ifootcornerR}[0]{Page \thepage}

\newcommand{\icode}[1]{\texttt{\footnotesize#1}}


\fontsizepteleven
\docsetup{End-to-End Instant Messaging Protocol Design Specification}{Neruthes}{\today}
\linespread{1.1}






\begin{document}

\fulldoctitle



\section*{Abstract}

This note describes a novel Instant Messaging (IM) protocol and the underlying Public Key Infrastructure (PKI) which enables the design.
The protocol features end-to-end encryption (E2EE), forward secrecy, and cryptographic identity (CID).



\Nmaketoc













\section{Background}

There have been popular IM protocols available and I have no intention to reinvent one from scratch.
However, security matters must be considered seriously.
I would like to slightly adjust mature designs so that my new protocol will make a firm response to the security matters of messaging.










\section{Design Principles}

I would like to achieve the following principles in this design.

\subsection{Security Principles}

\begin{compactitem}
	\item Undisclosed message signature for sender.
	\item Deniable message authentication.
	\item Native session management mechanism.
\end{compactitem}

\subsection{Usability Principles}

\begin{compactitem}
	\item Use SSH keypairs (Ed25519) for signing.
	\item Use OpenSSL keypairs (X25519) for encryption.
\end{compactitem}










\section{Existing Solutions}

Existing solutions include IM solutions and non-IM ones. This is a summary on why they are unsatisfactory in my perspective.

\begin{tabu}{|l|l|X|}
	\hline
	{Class}  & {Name}          & {Limitations}                                    \\
	\hline
	{IM}     & {Signal}        & {Requires phone number for user identification.} \\
	{ }      & {Telegram}      & {Requires phone number for user identification.} \\
	{ }      & {iMessage}      & {Precludes non-Apple devices.}                   \\
	{ }      & {uTox}          & {Cannot synchronize across devices.}             \\
	\hline
	{Non-IM} & {Email+OpenPGP} & {Heavy. No native session management.}           \\
	{ }      & {Email+S/MIME}  & {Like no reception at all?}                      \\
	\hline
\end{tabu}










\section{User Identification}

\subsection{Keypair Identification String}

Unlike account-based systems, our protocol believes that the public key itself is a ``quasi-account''.

\subsubsection{Definition}

Each user is identified by their Keypair Identification String (KIS), which looks like a domain name.
The KIS is a period-delimited string consisting of three components: the public key fingerprint,
the public key type identifier, and the namespace (\icode{pki}). The KIS should be case-insensitive.

For example:

\begin{lstlisting}
0a980cf5253c7efee51271989bf19c88.ssh-ed25519.pki
\end{lstlisting}

\subsubsection{Calculating Fingerprint}

Get the public key body (without header, footer, and trailing LF) and calculate its SHA-256 hash and get the initial 32 hexadecimal digits.

For example:

\begin{lstlisting}
printf "VvJceV08TFPnOWUwZpoSHMSzENgqyYuPI8H1RV" | sha256sum | cut -c1-32
\end{lstlisting}

\subsubsection{Calculating Pawprint}

Similar to Fingerprint, we need the concept ``Pawprint''.

The Pawprint is also calculated from the public key body.
Unlike the Fingerprint, it is the SHA-512 hash of the public key body (same definition as above) and is encoded in Base64.


For example:

\begin{lstlisting}
LQ327D41HbMrtBk651QXyinXrN7PgWrI8n8T7sqkzyOQtljgqK
uYE5E62zGXKOcHqCFGfZJpc8u7PywKzfBdrA==
\end{lstlisting}

(Linebreaks are for display purposes only and do not actually exist in the data stream.)










\section{Layering}

The message-exchanging operations may be divided into three layers.


\subsection{Relay Service Layer}

This layer cares how to relay encrypted messages between users.

% \subsubsection{Home Servers}
% \subsubsection{Handshaking}
% \subsubsection{Prioritization}
% \subsubsection{}

\subsection{Session Management Layer}

This layer cares how to maintain a persistent session for different scenarios.

\subsubsection{Private Session}

A private session is a bilateral session between two participants.

\subsubsection{Quasigroup Session}

A quasigroup session is a multi-party session among certain participants.

\subsubsection{Group Session}

A group session is a multi-party session consisting of dynamically changing participants.

\subsection{Message Layer (for Basic Messages)}

This layer cares how to make raw text pieces (and binary files) into structured message envelopes.

\subsubsection{Procedure Overview}

The whole procedure consists of the following steps:

\begin{compactitem}
    \item Construct raw message envelope according to message type.
    \item Add basic metadata (time, sessionId, senderId).
    \item Add extended metadata (replyTo, continueAt).
    \item Add message authentication codes (privSig, sharedSig).
    \item Construct final message envelope.
\end{compactitem}

\subsection{Message Layer (for Control Messages)}

This layer cares how to inform control sequences (e.g. audio call request, online status heartbeat) to session participants.








\section{Message Transportation}

// TODO










\section{Extra Optimizations}

// TODO










\end{document}
